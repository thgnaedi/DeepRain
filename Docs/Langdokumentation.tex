\documentclass[11pt,a4paper]{article}

% pdfLaTeX input and font settings
\usepackage[utf8]{inputenc}
\usepackage[T1]{fontenc}

\usepackage[english,ngerman]{babel}                 % German and English word separation settings
\newcommand{\gqq}[1]{\glqq #1\grqq{}}               % Own command for german quotes
\usepackage{graphicx}                               % Include graphics/images
\usepackage{float}
\usepackage{multirow}                               % Row spanning over multiple rows
\usepackage{subcaption}                             % Used for 2 images side-by-side
\usepackage{tabularx} % for 'tabularx' environment
\usepackage[bookmarks,                 % Creates bookmarks
            bookmarksnumbered=true,    % Numbers in PDF bookmarks
            bookmarksopen=true,        % Don't show bookmaks when opened
            hyperindex=true,           % Numbers in index are links
            pdfpagelabels=true,        % Set PDF page labels
            pdfa]{hyperref}                         % Klickable Links and ToC entries (should be last package !!!)

\hypersetup{
    pdfauthor    = {Thomas Gnaedig, Etienne Gramlich, Tim Hardenacke, Merle Wolf},
    pdftitle     = {DeepRain},
    pdfsubject   = {Deep Learning},
    pdfkeywords  = {Deep Learning,rain},
    pdflang      = {de},         % Set language of document to German
    pdfstartview = {FitH},       % Fit the document to screen horizontally (page width)
    pdfview      = {FitH},       % Fit the document to screen horizontally (page width)
    colorlinks   = {true},       % Use color for links
    linkcolor    = {black},      % Links in black (text color)
    urlcolor     = {black},      % URLs in black (text color)
    citecolor    = {black},      % Citations in black (text color)
    filecolor    = {black},      % File links in black (text color)
}

\title{DeepRain}
\author{Thomas Gnaedig, Etienne Gramlich, Tim Hardenacke, Merle Wolf}

\begin{document}
\maketitle
\tableofcontents

% Abstract
\begin{abstract}
Die Niederschlagvorhersage geschieht oft mit Modellen aus der Meteorologie, die auf physikalischen Gesetzen passieren. Unser Ziel war es zu bestimmen, ob der Niederschlag mit Methoden des Maschinellen Lernens vorhergesagt werden kann.

Dazu vergleichen wir verschiedene Typen von neuronalen Netzen und trainieren sie mit Daten des Deutschen Wetterdienstes und präsentieren die Ergebnisse und Genauigkeit für die Stadt Konstanz und ganz Deutschland.
\end{abstract}

\section{Daten}
Dieses Kapitel geht darauf ein, woher wir die Daten für das Training des Netzes beschaffen und wie wir diese vor-bearbeiten.

\subsection{Quelle}
Die verwendeten Daten stammen vom Deutschen Wetterdienst (ab jetzt DWD) und sind Radardaten von deren Radarstationen. Die Daten der Messstationen bilden Kreise um die jeweiligen Stationen und bedecken noch einen kleinen Bereich um Deutschland; die Daten geben die Stärke des Niederschlags an. Die Radar-Messungen liegen in einer Auflösung von einer Stunde und 5 Minuten vor, wir verwenden letztere. Z.zt. liegen die Daten vom Januar 2001 bis Januar 2018 vor (jeweils inklusive). Für unsere Zwecke verwenden wir die 18 kompletten Jahre 2001 bis 2017.\footnote{\url{https://opendata.dwd.de/climate\_environment/CDC/grids\_germany/5\_minutes/radolan/reproc/2017\_002/bin/}}

\subsubsection{Crawler}
Da die Daten in monatsweise in geschachtelten Archiven gepackt sind, haben wir einen Crawler geschrieben, der die Daten zuerst herunterlädt und auspackt. Mit den Kommandozeilenoptionen\footnote{\url{https://github.com/thgnaedi/DeepRain/tree/master/DWD_Crawler}} kann gesteuert werden, ob die stündlichen oder minütlichen Daten heruntergeladen werden und wohin die Binärdateien entpackt werden sollen. Wir empfehlen, die Binärdaten auf eine Btrfs\footnote{\url{https://en.wikipedia.org/wiki/Btrfs}}-formatierte Partition zu entpacken, da die Daten (wegen häufig auftretender Nullen bzw. kein Regen) leicht komprimierbar sind und die minütlichen Daten sonst mehr als ein Terabyte belegen würden.

\subsection{Preprocessing}
Um das Binärformat des DWD einzulesen, benötigt man die Python-Bibliothek \gqq{Wradlib}, die man über den Package-Manager von Anaconda installieren kann. Dann kann man die Datensätze als Deutschlandkarte mit einer Auflösung von 1100x900 Pixeln rastern.

Das Preprocessing geschieht in zwei Durchläufen: Zuerst wird das Maximum des Niederschlags bestimmt, das Minimum wird als 0 (kein Regen) angenommen. Danach werden die Werte auf einen Bereich gespreizt.

Beim zweiten Durchgang werden die Daten mithilfe des globalen Maximalwertes auf einen Wertebereich zwischen 0 und 255 umgerechnet, damit die Datensätze in einem Bildformat mit einer Bittiefe von 8 Bit gespeichert zur weiteren Verarbeitung werden können. Wir haben uns für das PNG-Format entschieden, weil es verlustfrei komprimiert und von platt­form­über­grei­fenden Bibliotheken gelesen und geschrieben werden kann.
Vor dem Speichern werden die Werte mit dem Faktor 4 multipliziert um Werte über 255 auf den Maximalwert abzuschneiden. Dadurch verwerfen wir Ausreißer und das Training funktioniert besser. Der Faktor 4 wurde empirisch bestimmt, weil das Berechnen eines Histogramms über alle 18 Jahre zu aufwendig gewesen wäre.

\subsubsection{Trainingsdaten}
\label{Samples}
Da sich unsere Aufgabenstellung mit einer Regenvorhersage für Konstanz befasst, sind die aktuell gespeicherten Bilder noch deutlich zu groß. Daher wird aus dem gespeicherten Bild nur ein kleiner Bereich um Konstanz herum ausgeschnitten. die Position von Konstanz wurde bereits zuvor in Kapitel \ref{locKN} bestimmt. Um diese Position wird nun ein Gebiet von 200x200 Pixeln extrahiert. Das so erzeugte Bild enthält dann alle Regenfronten, die das Wetter der nächsten 30 Minuten beeinflussen könnten. Um den Rechenaufwand zu reduzieren, werden die Bilder dann auf eine Größe von 64x64 Pixeln herab skaliert. Für eine Vorhersage haben wir uns dazu entschieden fünf Zeitschritte zu Verwenden. Es wird also das aktuelle Wetter, sowie die vergangenen 20 Minuten berücksichtigt.
Als Label dienen dann $n$ Zeitschritte, für den ersten Versuch sind $n = 1$, später soll auch weiter in die Zukunft vorhergesagt werden, hierzu werden $n = 7$ Zeitschritte verwendet.

\begin{figure}[h]
	\includegraphics[width=\linewidth]{pics/5Daten_1Label_Radar.png}
	\caption[Beispielhaftes Trainingssample zur vorhersage von 5 Minuten]{Die Grafiken timestep0 bis timestep4 sind die 5 eingehenden Daten, die als timestamp5 bezeichnete Grafik entspricht dem zu lernenden Label. Die Bilder sind jeweils 5 Minuten voneinander entfernte Radarbilder. Die orangene und rote Linie dient nur zur besseren Darstellung der Bewegung.}
	\label{5D1L}
\end{figure}

Die Daten für eine einfache fünf minutige Vorhersage sind in Abbildung \ref{5D1L} dargestellt.

Für das Training können allerdings nicht alle Daten verwendet werden. Zum einen, kommt es vor, dass eine Radarstation keine Daten liefert, solche Bilder eignen sich nicht für das Training. Auch kommt es sehr häufig vor, dass kein Regen stattfindet, die Grafik also komplett schwarz und ohne Struktur ist. Auch solche Samples sind nicht zum Training geeignet. Als letzte Einschränkung gilt, dass über alle fünf eingehenden Zeitschritte ein Mindestmaß an Regen zu sehen sein muss um in das Trainingsset aufgenommen zu werden. Für die Label gibt es keine Einschränkung, da sowohl die Fortbewegung von Regen, als auch das verschwinden gelernt werden soll. Jeder Zeitschritt wird maximal einmal in das Trainingsset aufgenommen, was einmal als Label verwendet wurde, wird keinesfalls in einem anderen Sample als als Eingabedatum verwendet. Beim Aufteilen des Sets zwischen Trainings und Validierungsset ist es wichtig, dass nicht zufällige Samples ausgewählt werden, damit garantiert ist, dass keine ähnliche Wetterlage bereits gesehen wurde.
Die so entstandenen Daten werden vor Eingabe in das Netz noch auf Werte zwischen 0 und 1 Normiert. Ab jetzt können die Daten für das Trainieren verwendet werden.



\subsection{Herausforderungen in diesem Kapitel}
Die größte Herausforderung in diesem Kapitel war zweifelsfrei die große Datenmenge, die wir verarbeitet haben. Die Rohdaten der 18 Jahre in 5-Minuten-Auflösung hätte unseren zugewiesenen Speicher gesprengt. Mit Btrfs, das die Dateien on-the-fly komprimiert und de-dupliziert, passten die Daten doch auf unsere Festplatte. Auch mussten wir eventuelle Ausreißer empirisch entfernen, weil das Berechnen des Histogramm über so viele Daten zu aufwendig wäre.

Darüber hinaus lagen die Archive des DWD im Tar-Format vor, das keine Checksumme anbietet und somit erst beim Entpacken bemerkt werden kann, dass beim Download einer Datei ein Fehler auftrat. Bei den großen Tar-Archiven des DWD ist leider mehrmals ein Download-Fehler aufgetreten, der sehr spät auffiel.


% Datenaufbereitung
\clearpage
\section{Daten-Aufbereitung}

\subsection{RADOLAN-Daten}

Die Daten des DWD werden über das Routineverfahren RADOLAN (Radar-Online-Aneichung) erfasst, dass durch eine Kombination von Niederschlagstationen und Wetterradaren hochauflösende Niederschlagsdaten produziert. 

Die Open-Source Bibliothek wradlib \footnote{\url{https://docs.wradlib.org/en/stable/index.html}} stellt für diese RADOLAN-Daten eine Stereographische Projektion zur Verfügung. Dies bedeutet das die Erdkugel zu einem Koordinatensystem aufgespannt wird, dessen Ursprung im Nordpol liegt. Siehe Abbildung  \ref{rz}.

\begin{figure}[H]
	\centering
	\includegraphics[width=0.8\textwidth]{pics/RZ_product.PNG}
	\caption{Ausschnitt Deutschlands im Koordinatensystem}
	\label{rz}
\end{figure}

Der resultierende Ausschnitt für Deutschland kann mit Hilfe von wradlib als 1100 * 900 Array ausgelesen werten, dessen Werte einer 1 Kilometer Grid-Box entsprechen.

\subsection{Location Konstanz}
Um für Konstanz und Umgebung sinnvolle Vorhersagen treffen zu können, muss der Standort von Konstanz auf den Daten markiert werden. Dazu müssen die Geographische Koordinaten für Konstanz in X- und Y-Koordinaten des Koordinatensystems umgewandelt werden. 

\begin{table}[H]
\begin{tabularx}{8cm}{|X|X|}
\hline
\multicolumn{2}{|l|}{ Geographische Koordinaten}   \\ \hline
         Latitude  & Longditude         \\ \hline
           47.66033 & 9.17582            \\ \hline
\end{tabularx} 	
\end{table}

\begin{table}[H]
\begin{tabularx}{8cm}{|X|X|}
\hline
\multicolumn{2}{|l|}{ XY - Koordinaten}   \\ \hline
        X-Koordinate & Y-Koordinate         \\ \hline
           -4602.6447 & -66.4622            \\ \hline
\end{tabularx} 	
\end{table}

Die umgewandelten XY-Koordinaten müssen nun innerhalb des Arrays gefunden werden. Dazu wurde über den Array iteriert und die Werte mit den errechneten Werten verglichen. Anschließend muss die Location auf der Karte markiert und dargestellt werden. Siehe Abbildung \ref{location}

\begin{figure}[H]
	\centering
	\includegraphics[width=0.8\textwidth]{pics/Location.png}
	\caption{Location von Konstanz im Ausschnitt Deutschland}
	\label{location}
\end{figure}

\subsection{Herausforderungen} 
Änderungen der Auflösung des Formats anpassen, richtige Datenquelle in wradlib finden. XY-Koordinaten innerhalb des Arrays finden und darstellen.

% Datenanalyse

% Netzwerke
\section{Netzwerkarchitekturen}
%erster Kontakt MNIST wie viel Trefferrate mit einfacher architektur
Für den ersten Kontakt mit einem Neuronalen Netz befassten wir uns mit dem MNIST Datensatz. Hierbei geht es um eine Klassifizierung von handschriftlichen Zahlen in die zugehörigen zehn Klassen. Um Rechenzeit zu sparen, da uns anfangs noch keine GPU zur Verfügung stand, haben wir das Netz sehr klein gehalten. Das Netz für welches wir uns entschieden haben, besteht aus einem Convolutional Layer mit 32 3x3 Kerneln, diese werden im späteren Kapitel~\ref{kapitelCNN} genauer erläutert. Anschließend folgt ein 2x2 MaxPooling Layer, auch darauf gehen wir später noch ein. Um die Klassifikation vornehmen zu können, wird anschließen ein Flatten Layer eingebunden und ein Fully-Connected Layer mit 128 Output-Neuronen. Diese 128 Features werden dann über ein weiteres Fully-Connected Layer auf die 10 Ausgabeklassen verrechnet. Als Aktivierungsfunktion dient ein softmax, welcher dafür sorgt, dass die Summe aller Ausgaben eins ergibt. Somit können die Outputs als Wahrscheinlichkeiten gedeutet werden, die wahrscheinlichste Klasse wird dann als Vorhersage verwendet.
Das hier erwähnte Fully-Connected Layer ist die einfachste Art eines Layers, es besteht aus mehreren Eingabeneuronen und einigen Ausgabeneuronen. Jedes Ausgabeneuron ist hierbei mit jedem Eingabeneuron verbunden. Um den Wert eines Ausgabeneurons zu berechnen wird dann einfach eine gewichtete Summe aller Eingaben berechnet. Die hierbei verwendeten Gewichte werden über die Fehlerfunktion gelernt.

%Aufgabenstellung bzw. Lösung
Für die kurzzeitige Wettervorhersage entschieden wir uns für zwei grundlegende Aufgabenstellungen. Die erste Herangehensweise war das Vorhersagen weiterer Radarbilder in der Zukunft. Die Architektur muss also mehrere zusammengehörende Radarbilder als Eingabe verarbeiten und als Ausgabe wieder ein oder mehrere Zeitschritte liefern. Für diese Aufgabenstellung eignet sich sowohl ein klassisches CNN (Kaptel \ref{kapitelCNN}), als auch ein UNet(Kaptel~\ref{kapitelUNet}). Um uns zwischen diesen Architekturen zu entscheiden nahmen wir einen kurzen Test vor, in welchem beide Architekturen mittels MSE einen Zeitschritt (5 Minuten) vorhersagen sollten.
\begin{figure}[h]
	\centering
	\includegraphics[width=\linewidth]{pics/Syntetische_Daten_CNN_UNet.png}
	\caption[Lernkurven verschiedener Architekturen auf synthetischen Daten]{Gezeigt ist die Lernkurve der Validierungsdaten auf 100 Epochen mit unterschiedlichen Netzarchitekturen. Die verwendeten Trainingsdaten sind 1000 synthetische Bilder welche eine wandernde Regenfront simulieren sollen. Das CNN ist etwa gleich gut wie das einfache UNet. Eine tiefere UNet Architektur (grün) erreicht eine noch bessere Performance. Die Architekturen sind in nachfolgenden Kapiteln genauer erläutert. }
	\label{imgCNNUNet}
\end{figure}

Die Abbildung~\ref{imgCNNUNet} zeigt die Lernkurve der beiden Architekturen auf die identische Problemstellung. Das UNet lernt in diesem Beispiel deutlich besser, weshalb die Vorhersage von Radarbildern in Zukunft mit einem UNet behandelt wird. Beide UNet Architekturen haben zu Beginn eine deutlich schlechtere Vorhersage als das CNN, bereits nach 60 Epochen ist die einfache UNet Architektur gleich gut, während das grüne UNet die Performance sogar übertrifft und nach 100 Epochen die besten Ergebnisse liefert.
\newline
Die zweite Herangehensweise ist eine Klassifikation, hierbei geht es nicht darum, das exakte Radarbild vorherzusagen, sondern einzuordnen, ob es regnet oder nicht. Diese Aufgabe wurde als einfache Klassifikation für Konstanz, als auch als pixelweise Klassifikation, für alle eingehenden Pixel durchgeführt. Für die Aufgabe der Klassifikation jedes Pixels wurde wieder ein UNet verwendet. Bei der Aufgabe der einfachen Klassifikation für Konstanz kamen beide Architekturen zum Einsatz.

\subsection{CNN}
\label{kapitelCNN}
Ein Convolutional Neural Networks im klassischen Sinne ist ein Netzwerk mit mehreren Convolutional-Layer, häufig in Verbindung mit Pooling-Layer. Ein Convolutional-Layer besteht aus mehreren Filtern. Die Filter berechnen ein Output in Abhängigkeit mehrerer benachbarter 'Pixel'; die Größe der berücksichtigten Region hängt von der Filter- bzw. Kernelgröße ab.
Vorteile von Convolutional Layers sind zum einen, dass Nachbarschaften berücksichtigt werden, was gerade bei Bildern sehr sinnvoll ist, aber auch, dass der Speicherbedarf sehr gering ist, da nur eine kleine Anzahl an Gewichten für das komplette Bild verwendet werden müssen.
Ein Pooling-Layer reduziert die Feature-Größe indem jeweils nur das stärkste Signal einer Region weitergegeben werden.

%MSE:
Für die Vorhersage der Radardaten haben wir uns für ein sehr einfaches CNN (Convolutional Neural Network) entschieden.
Als Eingabe erwartet es mehrere Zeitschritte um darauf eine Vorhersage zu treffen. Der Input ist also dreidimensional wobei die dritte Dimension die Zeitschritte und die anderen Dimensionen die Bildauflösung beschreiben. Die Eingabe wird durch sechzehn Convolution Kernel der Größe 5x5 verrechnet. Anschließend folgen zweiunddreißig weitere 5x5 Kernel. Des weiteren ist ein optionales Dropout Layer eingebunden, welches nur zu Testzwecken verwendet wurde. Die Performance hatte sich dadurch aber nicht bemerkenswert verändert. Abschließend kommt ein Kernel, welcher die nun entstandenen Features zu einem Bild zusammenfasst. Hierzu wird ein 3x3 Kernel verwendet. Alle Layer sind mit einer ReLu als Aktivierungsfunktion ausgestattet.
Die Performance ist weniger gut, als ein UNet kann aber mit der sehr einfachen UNet Architektur mithalten. Zu sehen ist das in Abbildung \ref{imgCNNUNet} wo die blaue Kurve dem Lernverhalten der hier beschriebenen CNN Architektur entspricht.

%Klassifikation:
Für Klassifikation des Konstanz-Pixels entschlossen wir uns dazu, unser ursprüngliches CNN zum Lernen des MNIST Datensatz wiederzuverwenden. Dieses Netz besteht lediglich aus zweiunddreißig 3x3 Kerneln mit ReLu Aktivierungsfunktion. Anschließend folgt ein 2x2 MaxPooling und ein Flatten Layer. Die nun \gqq{flachen} Daten werden durch ein FullyConnected Layer auf 30 Neuronen reduziert. Darauf folgt ein Dropout Layer mit 20\% welches Overfitting verhindern soll. Abschließend folgt ein Layer, dass die Features auf drei Ausgabe-Neuronen verrechnet. Damit die Klassifizierung einfacher zu interpretieren ist, wird als Aktivierungsfunktion ein SoftMax verwendet.

\subsection{UNet}
\label{kapitelUNet}
%was ist ein Unet?
Ein Unet ist eine spezielle Form eines CNN. Es hat seinen Namen durch die 'U-Förmige' Architektur (siehe Abbildung~\ref{imgUNetA}).
\begin{figure}[h]
	\centering
	\includegraphics[width=\linewidth]{pics/UNet_Biomedical}
	\caption[UNet aus Paper von O. Ronneberger, P. Fischer und T. Brox]{Die hier abgebildete Architektur entstammt dem Paper zur biomedizinischen Bildsegmentierung \cite{DBLP:journals/corr/RonnebergerFB15}. Hier abgebildet ist eine typische UNet-Architektur, welche von den Eingabebildern (oben links) zu der Ausgabe (oben rechts) die Features stets verkleinert (rote Pfeile) und im späteren Verlauf auch wieder vergrößert (grüne Pfeile). Da bei dem wieder vergrößern der Daten nicht alle Informationen wiedergewonnen werden können, gibt es auch Querverbindungen (graue Pfeile) welche die zuvor errechneten Features an ein späteres Layer weitergibt.}
	\label{imgUNetA}
\end{figure}

Das Unet besteht aus mehreren Convolutional Layern, mit anschließendem Pooling. Hierdurch wird die Featuregröße immer weiter reduziert. Ab einem bestimmten Punkt wird die Featuregröße wieder aufgeblasen, um am Ende für den Output die gleiche Größe wie für den Input zu besitzen. Beim 'Aufblasen' kommt ein Upsampling Layer zum Einsatz. Da allerdings beim Vergrößern der Features nicht alle Informationen wiederhergestellt werden können, gibt es auch noch horizontale Verbindungen, durch welche zuvor erstellte Features später weiterverwendet werden können.
Diese Architektur ist sehr ähnlich zu einem Autoencoder und wurde erstmals für biomedizinische Zwecke verwendet. Da sich mit dieser Architektur mehrere Bilder einlesen und auch beliebig viele Bilder ausgeben lassen, versuchen wir damit die Wettervorhersage für mehrere Zeitschritte zu lösen.

%Einstiegsnetz
Um eine geeignete Architektur auszuwählen haben wir ein einfaches Testszenario aufgebaut, welches unterschiedliche Netzarchitekturen zu lösen hatten. Die hierfür verwendeten Daten waren aus jeweils 100x100x5 Pixeln großen synthetischen Daten aufgebaut.
%ToDo: Synthetische Daten erklären.
Das erste U-Net besteht aus lediglich einem Upsampling Layer. Die Architektur sieht wie folgt aus:
Anfangs verarbeiten zweiunddreißig 5x5 Convolutional Layer mit Padding die Eingabe. Diese wird über ein 3x3 MaxPooling reduziert. Die reduzierten Daten werden durch ein 3x3 Upsampling wieder in die ursprüngliche Größe zurück gewandelt und mit den Features vor dem MaxPooling verbunden. Die so entstehenden 100x100x64 Features werden anschließend über einen 1x1 Convolutional Layer zu einem Ausgabebild zusammengefasst.
Die zu diesem Netz gehörende Lernkurve ist in Abbildung~\ref{imgCNNUNet} zu sehen. Die Performance unterscheidet sich kaum zu der eines klassischen CNNs. Daher versuchen wir die Architektur tiefer zu gestalten.

%Unet64
Das finale UNet wie es in Abbildung \ref{imgCNNUNet} in grün zu sehen ist besteht aus mehreren Ebenden.
Zunächst wird die Eingabe durch ein Convolutional-Layer mit zehn 3x3 Kerneln und anschließender ReLU als Aktivierungsfunktion geleitet. Die so entstehenden Features werden als Conv01 bezeichnet und später wieder verwendet.
Die Features werden anschließend durch ein MaxPooling-Layer der Größe 2x2 verkleinert. Anschließend folgt ein weiteres Convolutional-Layer mit zwanzig 3x3 Kerneln. diese Features werden im weiteren als Conv02 bezeichnet.
Auch hierauf folgt wieder ein 2x2 MaxPooling-Layer mit anschließend einem weiteren Convolutional-Layer mit zwanzig Kerneln der Größe 3x3. Diese Features werden als Conv03 weiterverwendet.
Die tiefste Ebene in dieser Architektur besteht wieder aus einem Convolutional-Layer mit zwanzig Kerneln der Größe 3x3. Auch hier folgt ein 2x2 MaxPooling. Diese Features heißen Conv04.
Ab jetzt werden die Features wieder größer. Ein Upsampling-Layer der Größe 2x2 sorgt dafür, dass die Features mit den zuvor erstellten Conv04 Features verbunden werden können. Dies geschieht durch ein Concatenate-Layer, was die beiden Tensoren zu einem Größeren verbindet. Die so entstandenen Features werden wieder durch ein Upsampling-Layer der größe 2x2 vergrößert und mit den Conv03 Features verbunden. Auch diese Daten werden wieder über ein Upsampling-Layer der größe 2x2 verrechnet und anschließend mit den Conv02 Features verbunden. Abschließend werden die Daten durch ein weiteres 2x2 Upsampling-Layer verrechnet und mit den Conv01 Features verbunden. Abschließend kommt ein als Ausgabe ein Convolutional-Layer mit einem 1x1 Kernel der alle Features zu einem Graustufenbild zusammenfasst.
Alle Hier erwähnten Convolutional-Layer haben eine ReLU als Aktivierungsfunktion, wodurch negative Werte ausgeschlossen werden. Des weiteren besitzen diese Layer auch Padding um die Bildgröße beizubehalten.

Alle oben genannten Architekturen werden mit dem Adam-Optimizer trainiert. Je nach Aufgabenstellung dient als Fehlerfunktion der MSE, beim Vorhersagen der Radardaten, oder die Categorical Cross-Entropy, beim Lösen des Klassifikationsproblems.

\section{Klassifizierung}
Statt die genaue Regenmenge vorherzusagen, stellten wir drei Kategorien auf: kein Regen ($= 0mm$), wenig Regen ($\leq 8mm$) und viel Regen ($> 8mm$). Diese Kategorien haben wir als One-Hot-Vector kodiert. `[1, 0, 0]` entspricht hierbei kein Regen, sodass man aus der ersten Dimension der Vorhersage einfach ein Vorschaubild generieren kann aus dem man gleich feststellen kann, ob es am jeweiligen Pixel regnet oder nicht.

Für das Training mit Kategorien kann man nicht mehr den MSE verwenden, hier würde selbst nach 80 Epochen nur "kein Regen" vorhergesagt. Stattdessen wurde als Loss-Funktion die \enquote{Categorical Crossentropy} von Keras verwendet; die binäre Crossentropy können wir nicht verwenden, weil wir mehr als zwei Kategorien verwenden. Die \enquote{Categorical Crossentropy} funktioniert relativ gut, aber es wird ein Blob vorhersagt, der etwas über den Bereich ragt, in dem es eigentlich regnet.

Danach wurde noch die Aktivierungsfunktion für den Output-Layer Sigmoid durch Softmax ersetzt. Dadurch erscheint das Vorschaubild etwas verwaschener, aber der Blob um das Regengebiet wird kleiner und die Differenz zum Referenzbild wird kleiner.

Wenn man die Aktivierungsfunktion der Hidden-Layer (von ReLu) zu Tanh verändert, verbessert sich auch die Kategorisierung: der Blob nähert sich weiter dem Regengebiet aus dem zu vorhersagendem Bild an, ist aber immer noch merkbar größer und franst an den kanten aus.

Als nächstes wird die Metrik "categorical\_accuracy" verwendet, um die Vorhersage zu überwachen. Dadurch kann der Fortschritt beim Trainieren besser überwacht werden.


\subsection{Training}
Für das Training wurden alle Daten der 18 Jahre verwendet, partitioniert in Trainings- und Evaluationsdaten. Als Lossfunktion wurde der MSE verwendet, der den Unterschied über alle Pixel kumuliert, weswegen so hohe Werte in den Abbildungen vorkommen (1100x900 Pixel). Da das Training auf der GPU weniger al 10~Sekunden pro Epoche dauert, wurden gleich 3072 Epochen trainiert.

In Abbildung~\ref{fig:lernkurven} sind die Lernkurven des Trainings nebeneinander dargestellt, links steht das Training des Netzes mit Softmax als Aktivierungsfunktion des Hidden Layer und rechts ist das selbe Netz mit TanH als Aktivierungsfunktion. Man sieht jedoch, dass es ab etwa 1500~Epochen (x-Achse) keine Verbesserungen mehr gibt.

\begin{figure}[ht]
\centering
\begin{subfigure}{0.5\textwidth}
\centering
\includegraphics[width=\linewidth]{pics/lernkurve_activationHidden-softmax_activationOutput-softmax}
\caption{Lernkurve (Hidden layer: Softmax)}
\label{fig:lernkurveSoftmax}
\end{subfigure}%
\begin{subfigure}{0.5\textwidth}
\centering
\includegraphics[width=\linewidth]{pics/lernkurve_activationHidden-tanh_activationOutput-softmax}
\caption{Lernkurve (Hidden layer: Tanh)}
\label{fig:lernkurveTanh}
\end{subfigure}%
\caption{Lernkurven von den Aktivierungsfunktionen der Hidden-Layer}
\label{fig:lernkurven}
\end{figure}



\subsection{Auswertung}
Zu den Netzen mit den verschiedenen Aktivierungsfunktionen wurden jeweils eine Confustion-Matrix stellt (siehe Tabellen~\ref{tab:confusionSoftmax} und~\ref{tab:confusionTanh}). An beiden Matrizen kann man sehen, dass beide Aktivierungsfunktionen für einzelne Pixel statistisch ähnliche Ergebnisse liefern.

Bei beiden Matrizen ist gleich, dass, falls kein Regen vorhergesagt wird, in 96,1\% respektive 96,3\% auch tatsächlich kein Regen eintrifft. Wird wenig Regen vorhergesagt, ist die Unsicherheit recht groß: zu rund 20\% regnet es gar nicht oder zu etwa 16\% regnet es stark. Zu einem Fünftel kann es also sein, dass hier die Vorhersage \enquote{Regen} nicht eintrifft. Wird starker Regen vorausgesagt, trifft zu gut 68\% auch starker Regen ein, zu fast 26\% wenig Regen oder gut 5\% kein Regen. Hier trifft die Vorhersage \enquote{Regen} also zu 95\% ein.

\begin{table}[ht]
\centering
\begin{tabular}{lr|rrr}
    &                      & \multicolumn{3}{c}{Vorhersage}\\
    &                      & \textbf{Kein Regen} & \textbf{Wenig Regen} & \textbf{Viel Regen}\\\hline
\multirow{3}{*}{\rotatebox{90}{Daten}}
    & \textbf{Kein Regen}  & 2227245 (96,1\%)    & 33383 (19,9\%)       & 3762 (05,7\%)\\
    & \textbf{Wenig Regen} & 81116 (03,5\%)      & 106434 (63,4\%)      & 17077 (25,9\%)\\
    & \textbf{Viel Regen}  & 9695 (00,4\%)       & 27930 (16,7\%)       & 45166 (68,4\%)\\
\end{tabular}
\caption{Confustion-Matrix (Aktivierungsfunktion Hidden Layer: Softmax)}
\label{tab:confusionSoftmax}
\end{table}

\begin{table}[ht]
\centering
\begin{tabular}{lr|rrr}
    &                      & \multicolumn{3}{c}{Vorhersage}\\
    &                      & \textbf{Kein Regen} & \textbf{Wenig Regen} & \textbf{Viel Regen}\\\hline
\multirow{3}{*}{\rotatebox{90}{Daten}}
    & \textbf{Kein Regen}  & 2225229 (96,3\%)    & 35527 (20,4\%)       & 3634 (05,4\%)\\
    & \textbf{Wenig Regen} & 76988 (03,3\%)      & 110399 (63,5\%)      & 17240 (25,8\%)\\
    & \textbf{Viel Regen}  & 8849 (00,4\%)       & 27969 (16,1\%)       & 45973 (68,8\%)\\
\end{tabular}
\caption{Confustion-Matrix (Aktivierungsfunktion Hidden Layer: Tanh)}
\label{tab:confusionTanh}
\end{table}

Bei der Vorhersage von zusammenhängenden Niederschlagsmengen auf einer Karte gibt es Unterschiede zwischen den Aktivierungsfunktionen der Hidden Layer, während es auf die Wahrscheinlichkeit der einzelnen Pixel kaum unterschiede gab.
Das UNet mit TanH als Aktivierungsfunktion der Hidden Layer erzeugt größere (und leicht rundere) Blobs (Abbildung~\ref{fig:hiddenActivationTanh}). Bei Softmax werden schräge Kanten besser vorhergesagt (Abbildung~\ref{fig:hiddenActivationSoftmax}).
\begin{figure}[ht]
\centering
\begin{subfigure}{0.5\textwidth}
\centering
\includegraphics[width=\linewidth]{pics/categorical_crossentropy_hidden-softmax_output-softmax_above_3072}
\caption{Hidden layer activation: Softmax}
\label{fig:hiddenActivationSoftmax}
\end{subfigure}%
\begin{subfigure}{0.5\textwidth}
\centering
\includegraphics[width=\linewidth]{pics/categorical_crossentropy_hidden-tanh_output-softmax_above_3072}
\caption{Hidden layer activation: Tanh}
\label{fig:hiddenActivationTanh}
\end{subfigure}%
\caption{Vergleich von Aktivierungsfunktionen der Hidden-Layer}
\label{fig:activatinHidden}
\end{figure}


\subsubsection{Zwei Kategorien}
Daraufhin versuchten wir die Vorhersage mit nur zwei Kategorien (Regen / Kein Regen) durchzuführen um festzustellen, ob die Vorhersage ob es regnet oder nicht besser funktioniert als die mit keinem, wenig oder starkem Regen. Dazu erstellen noch einmal die Confusion Matrix auf (siehe Tabelle~\ref{tab:confusionTwoCategoriesTresholdZero}). Bei einem Treshold von 0 erhält man fast gleiche Wahrscheinlichkeiten wie bei 3 Kategorien für Regen und Kein Regen: Falls kein Regen vorhergesagt wurde, regnet es zu 96\% nicht; und falls Regen vorhergesagt wurde regnet es zu gut 80\%.

\begin{table}[ht]
\centering
\begin{tabular}{lr|rr}
    &                      & \multicolumn{2}{c}{Vorhersage}\\
    &                      & \textbf{Kein Regen} & \textbf{Regen}\\\hline
\multirow{3}{*}{\rotatebox{90}{Daten}}
    & \textbf{Kein Regen}  & 2218754 (96,3\%)    & 45636 (18,4\%)\\
    & \textbf{Regen}       & 85427 (03,7\%)      & 201991 (81,6\%)\\
\end{tabular}
\caption{Confustion-Matrix (Zwei Kategorien, Treshold: 0)}
\label{tab:confusionTwoCategoriesTresholdZero}
\end{table}

\begin{table}[ht]
\centering
\begin{tabular}{lr|rr}
    &                      & \multicolumn{2}{c}{Vorhersage}\\
    &                      & \textbf{Kein Regen} & \textbf{Regen}\\\hline
\multirow{3}{*}{\rotatebox{90}{Daten}}
    & \textbf{Kein Regen}  & 2272947 (96,7\%)    & 39276 (19,6\%)\\
    & \textbf{Regen}       & 78475 (03,3\%)      & 161110 (80,4\%)\\
\end{tabular}
\caption{Confustion-Matrix (Zwei Kategorien, Treshold: 2)}
\label{tab:confusionTwoCategoriesTresholdTwo}
\end{table}



\subsection{Herausforderungen in diesem Kapitel}
\begin{itemize}
\item Richtige Kategorien finden
\item Training mit richtiger Aktivierungsfunktion / Optimizer
\end{itemize}


% Ziel und Test: MSE

% Fazit / Ausblick

\newpage
\listoffigures
\listoftables

\end{document}
