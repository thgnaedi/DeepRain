\documentclass[11pt,a4paper]{article}

% pdfLaTeX input and font settings
\usepackage[utf8]{inputenc}
\usepackage[T1]{fontenc}
\usepackage{amsmath}

\usepackage[english,ngerman]{babel}                 % German and English word separation settings
\newcommand{\gqq}[1]{\glqq #1\grqq{}}               % Own command for german quotes
\usepackage{graphicx}                               % Include graphics/images
\usepackage{float}
\usepackage{multirow}                               % Row spanning over multiple rows
\usepackage{subcaption}                             % Used for 2 images side-by-side
\usepackage{tabularx}                               % for 'tabularx' environment
\usepackage[font={small}]{caption}
\usepackage[printonlyused,withpage]{acronym}        % Abkürzungsverzeichnis, https://de.wikibooks.org/wiki/LaTeX-W%C3%B6rterbuch:_Abk%C3%BCrzungsverzeichnis
\usepackage[bookmarks,                 % Creates bookmarks
            bookmarksnumbered=true,    % Numbers in PDF bookmarks
            bookmarksopen=true,        % Don't show bookmaks when opened
            hyperindex=true,           % Numbers in index are links
            pdfpagelabels=true,        % Set PDF page labels
            pdfa]{hyperref}                         % Klickable Links and ToC entries (should be last package !!!)

\hypersetup{
    pdfauthor    = {Thomas Gnaedig, Etienne Gramlich, Tim Hardenacke, Merle Wolff},
    pdftitle     = {DeepRain},
    pdfsubject   = {Deep Learning},
    pdfkeywords  = {Deep Learning,rain},
    pdflang      = {de},         % Set language of document to German
    pdfstartview = {FitH},       % Fit the document to screen horizontally (page width)
    pdfview      = {FitH},       % Fit the document to screen horizontally (page width)
    colorlinks   = {true},       % Use color for links
    linkcolor    = {black},      % Links in black (text color)
    urlcolor     = {black},      % URLs in black (text color)
    citecolor    = {black},      % Citations in black (text color)
    filecolor    = {black},      % File links in black (text color)
}

\title{DeepRain}
\author{Thomas Gnaedig, Etienne Gramlich, Tim Hardenacke, Merle Wolf}

\begin{document}
\maketitle
\tableofcontents
\newpage

\begin{acronym}[DWD] % Hier steht die längste Abkürzung, damit die Breite stimmt
 \acro{DWD}{Deutscher Wetterdienst}
 \acro{MSE}{Mean Squared Error}
\end{acronym}

% Abstract
\begin{abstract}
Die Niederschlagvorhersage geschieht oft mit Modellen aus der Meteorologie, die auf physikalischen Gesetzen passieren. Unser Ziel war es zu bestimmen, ob der Niederschlag mit Methoden des Maschinellen Lernens vorhergesagt werden kann.

Dazu vergleichen wir verschiedene Typen von neuronalen Netzen und trainieren sie mit Daten des Deutschen Wetterdienstes und präsentieren die Ergebnisse und Genauigkeit für die Stadt Konstanz und ganz Deutschland.~\cite{DBLP:journals/corr/RonnebergerFB15}
\end{abstract}

%Einleitung zu DeepLearning
\section{Abstract}
Das Teamprojekt DeepRain bestehend aus fünf Studierenden der Masterstudiengänge MSI und BIT startete im Wintersemester 18/19 an der HTWG Konstanz. Das Ziel Projekts ist, innerhalb eines Jahres einen lauffähigen Algorithmus auf Github (thgnaedi/DeepRain) zur Verfügung zu stellen. Dieser soll mit Hilfe von Deep Learning in der Lage sein das Wetter, bzw. den Niederschlag, in Konstanz über den Zeitraum von einer Stunde vorherzusagen. Das einjährige Projekt wird von Prof. Oliver Duerr betreut. 
Eine möglichst genaue Wettervorhersage hat viele offensichtliche Vorteile. Beginnend mit dem täglichen Verlassen des Hauses und der Entscheidung; mit oder ohne den Regenschirm? Vermutlich treffen die meisten diese Entscheidung basierend auf der Wettervorhersage und vertrauen darauf, dass diese auch korrekt ist.
Vorhersagen mit Deep Learning-Unterstützung finden auch eine sehr wichtige Verwendung in der Katastrophenvorhersage und dem Katastrophenmanagement (vgl. \cite[S. 763]{Hanif.2019}). Eine weitere wichtige Rolle spielt es bei der Energieversorgung, beispielsweise bei der Planung der Auslastung von Solar-Panels (vgl. \cite[S. 2]{AndreGensleret.al..}). Die eigentliche Wettervorhersage wird jedoch bisher noch nicht mit Deep Learning-Techniken generiert, dies ist erst ab diesem Jahr bei einigen Anbietern angedacht (vgl. \cite{ChristophFrohlich.2019}. 
\begin{figure}[ht]
\centering
\includegraphics[width=\linewidth]{pics/Deep_learning_prozess}
\caption{Wichtige Schritte im Prozess des DeepRain-Projekts, eigene Darstellung}
\label{fig:deepLearningProcess}
\end{figure}
Im Rahmen des Teamprojekts stand zu Beginn ein einarbeiten in die Prozesse des Maschine-Learnings an, sowie den Stand der Technik zu recherchieren. Daraufhin stand die Datenbeschaffung im Vordergrund und die Qualitätsbewertung der vorhandenen Daten. Im späteren Verlauf des Projekts ging es dann an die Entwicklung des Algorithmus. Der Ablauf dieser Arbeit lehnt sich an diesen, auch in Abbildung~\ref{fig:deepLearningProcess} dargestellten, Prozess an. 

\subsection{Deep Learning}
Deep Learning basiert auf der Optimierung von künstlichen neuronalen Netzen (KNN) und ist eine Weiterentwicklung des Maschine Learning (vgl. \cite[S. 1]{Georgevici.2019}). Ziel von neuronalen Netzen ist es, eine ähnliche Lösungsweise zu ermöglichen, wie im menschlichen Gehirn. Kern ist hier das “Lernen” und das Lösen von Problemen. Neuronale Netze haben sich vor allem wegen ihrer Fähigkeit der Verarbeitung von ständig wachsenden Datenmengen und -komplexität etabliert (vgl. \cite[S. 373]{Welsch.2018}). In einem künstlichen neuronalen Netzwerk müssen im Vorhinein keine Vermutungen über Zusammenhänge festgelegt werden, denn diese werden während des Lernprozesses vom Netz ermittelt (vgl. \cite[S. 581]{Backhaus.2018b}). Ein Neuronales Netz ist 
so aufgebaut, dass es einen bekannten Input gibt, welcher durch einen sogenannten Hidden-Layer läuft. Der Hidden-Layer besteht aus Neuronen in denen das Lernen stattfindet und schließlich ein Output generiert wird. Dabei können einige Variablen, die den Output beeinflussen, festgelegt werden. 
Die Neuronen modifizieren dabei die Daten und leiten diese anschließend weiter (vgl. \cite[S. 373]{Welsch.2018}). Es kann sich auch zwischen einer biologischen oder einer technischen Simulation von Neuronen entschieden werden (vgl. \cite{https:www.facebook.comspektrumverlag.04.12.2014}). Die technische Simulation wird zur Datenverarbeitung und Mustererkennung meist bevorzugt (vgl. \cite{https:www.facebook.comspektrumverlag.04.12.2014}). Dieser Zusammenhang wird in Abbildung~\ref{fig:NeuralVsDeepNeural} dargestellt. 
In dieser Abbildung wird auch ersichtlich, wodurch sich ein tiefes neuronales Netzwerk von einem neuralen Netz unterscheidet. Hier wird gleich mit mehreren Hidden-Layers, die hintereinander liegen, gearbeitet. Die Verwendung mehrerer Hidden-Layers führt dabei laut (\cite[S. 581]{Backhaus.2018b}) zu erfahrungsgemäß korrekteren Lernergebnissen. Das „Lernen“ hängt dann vom Grad der Aktivierung der einzelnen Neuronen ab. Dieser wird durch die bisherigen Ergebnisse durch eine ermittelte Gewichtung einzelner Neuronen bestimmt. Diese Gewichtung wird im Verlauf des Lernens im Netz im besten Fall immer weiter optimiert (vgl. \cite[S. 586]{Backhaus.2018b}).
\begin{figure}[ht]
\centering
\includegraphics[width=\linewidth]{pics/ANN_for_deep_learning_P35}
\caption{Unterscheidung Neurales Netzwerk und einem mehrstufigen neuronalem Netzwerk, Quelle: ANN for deep learning (Akerkar 2019, S. 35)}
\label{fig:NeuralVsDeepNeural}
\end{figure}
In unserem Projekt soll mithilfe von neuronalen Netzen, das Lernen aus bisherigen Wetterdaten ermöglicht werden, sodass eine möglichst korrekte Wettervorhersage der kommenden Stunde generiert wird.

\subsection{State-of-the-art Wettervorhersage mit Deep Learning}
Obwohl für Deep Learning, Machine-Learning, sowie Neuro- und Lingual-Programming einiges an Literatur zu finden ist, steht die Nutzung zur Wettervorhersage noch am Anfang (vgl. \cite{ChristophFrohlich.2019}). So gibt es auf der Plattform GitHub zum aktuellen Stand (August 2019) gerade einmal eine Handvoll Repositories zum Thema Wettervorhersage mit Deep Learning. Verfahren werden verbessert, die Lernverfahren den Menschlichen Lernen immer ähnlicher, die Entwicklung ist stetig steigend und wie bereits erwähnt sind noch viele Bereiche wie u.a. die Wettervorhersage  ergründbar (\cite[S. 103]{Wick.2017}). Im Bereich der Wettervorhersage bietet sich, in Anlehnung an Abbildung 3, in den meisten Fällen das Supervised Learning an, da eine möglichst genaue Vorhersage erreicht werden soll. Aber auch das  nicht-überwachte Lernen kann interessante Schlüsse zulassen, um eventuell unbekannte Muster zu finden (vgl. \cite[S. 371]{Welsch.2018}). 


\section{Daten}
Dieses Kapitel geht darauf ein, woher wir die Daten für das Training des Netzes beschaffen und wie wir diese vor-bearbeiten.

\subsection{Quelle}
Die verwendeten Daten stammen vom Deutschen Wetterdienst (ab jetzt DWD) und sind Radardaten von deren Radarstationen. Die Daten der Messstationen bilden Kreise um die jeweiligen Stationen und bedecken noch einen kleinen Bereich um Deutschland; die Daten geben die Stärke des Niederschlags an. Die Radar-Messungen liegen in einer Auflösung von einer Stunde und 5 Minuten vor, wir verwenden letztere. Z.zt. liegen die Daten vom Januar 2001 bis Januar 2018 vor (jeweils inklusive). Für unsere Zwecke verwenden wir die 18 kompletten Jahre 2001 bis 2017.\footnote{\url{https://opendata.dwd.de/climate\_environment/CDC/grids\_germany/5\_minutes/radolan/reproc/2017\_002/bin/}}

\subsubsection{Crawler}
Da die Daten in monatsweise in geschachtelten Archiven gepackt sind, haben wir einen Crawler geschrieben, der die Daten zuerst herunterlädt und auspackt. Mit den Kommandozeilenoptionen\footnote{\url{https://github.com/thgnaedi/DeepRain/tree/master/DWD_Crawler}} kann gesteuert werden, ob die stündlichen oder minütlichen Daten heruntergeladen werden und wohin die Binärdateien entpackt werden sollen. Wir empfehlen, die Binärdaten auf eine Btrfs\footnote{\url{https://en.wikipedia.org/wiki/Btrfs}}-formatierte Partition zu entpacken, da die Daten (wegen häufig auftretender Nullen bzw. kein Regen) leicht komprimierbar sind und die minütlichen Daten sonst mehr als ein Terabyte belegen würden.

\subsection{Preprocessing}
Um das Binärformat des DWD einzulesen, benötigt man die Python-Bibliothek \gqq{Wradlib}, die man über den Package-Manager von Anaconda installieren kann. Dann kann man die Datensätze als Deutschlandkarte mit einer Auflösung von 1100x900 Pixeln rastern.

Das Preprocessing geschieht in zwei Durchläufen: Zuerst wird das Maximum des Niederschlags bestimmt, das Minimum wird als 0 (kein Regen) angenommen. Danach werden die Werte auf einen Bereich gespreizt.

Beim zweiten Durchgang werden die Daten mithilfe des globalen Maximalwertes auf einen Wertebereich zwischen 0 und 255 umgerechnet, damit die Datensätze in einem Bildformat mit einer Bittiefe von 8 Bit gespeichert zur weiteren Verarbeitung werden können. Wir haben uns für das PNG-Format entschieden, weil es verlustfrei komprimiert und von platt­form­über­grei­fenden Bibliotheken gelesen und geschrieben werden kann.
Vor dem Speichern werden die Werte mit dem Faktor 4 multipliziert um Werte über 255 auf den Maximalwert abzuschneiden. Dadurch verwerfen wir Ausreißer und das Training funktioniert besser. Der Faktor 4 wurde empirisch bestimmt, weil das Berechnen eines Histogramms über alle 18 Jahre zu aufwendig gewesen wäre.

\subsubsection{Trainingsdaten}
\label{Samples}
Da sich unsere Aufgabenstellung mit einer Regenvorhersage für Konstanz befasst, sind die aktuell gespeicherten Bilder noch deutlich zu groß. Daher wird aus dem gespeicherten Bild nur ein kleiner Bereich um Konstanz herum ausgeschnitten. die Position von Konstanz wurde bereits zuvor in Kapitel \ref{locKN} bestimmt. Um diese Position wird nun ein Gebiet von 200x200 Pixeln extrahiert. Das so erzeugte Bild enthält dann alle Regenfronten, die das Wetter der nächsten 30 Minuten beeinflussen könnten. Um den Rechenaufwand zu reduzieren, werden die Bilder dann auf eine Größe von 64x64 Pixeln herab skaliert. Für eine Vorhersage haben wir uns dazu entschieden fünf Zeitschritte zu Verwenden. Es wird also das aktuelle Wetter, sowie die vergangenen 20 Minuten berücksichtigt.
Als Label dienen dann $n$ Zeitschritte, für den ersten Versuch sind $n = 1$, später soll auch weiter in die Zukunft vorhergesagt werden, hierzu werden $n = 7$ Zeitschritte verwendet.

\begin{figure}[h]
	\includegraphics[width=\linewidth]{pics/5Daten_1Label_Radar.png}
	\caption[Beispielhaftes Trainingssample zur vorhersage von 5 Minuten]{Die Grafiken timestep0 bis timestep4 sind die 5 eingehenden Daten, die als timestamp5 bezeichnete Grafik entspricht dem zu lernenden Label. Die Bilder sind jeweils 5 Minuten voneinander entfernte Radarbilder. Die orangene und rote Linie dient nur zur besseren Darstellung der Bewegung.}
	\label{5D1L}
\end{figure}

Die Daten für eine einfache fünf minutige Vorhersage sind in Abbildung \ref{5D1L} dargestellt.

Für das Training können allerdings nicht alle Daten verwendet werden. Zum einen, kommt es vor, dass eine Radarstation keine Daten liefert, solche Bilder eignen sich nicht für das Training. Auch kommt es sehr häufig vor, dass kein Regen stattfindet, die Grafik also komplett schwarz und ohne Struktur ist. Auch solche Samples sind nicht zum Training geeignet. Als letzte Einschränkung gilt, dass über alle fünf eingehenden Zeitschritte ein Mindestmaß an Regen zu sehen sein muss um in das Trainingsset aufgenommen zu werden. Für die Label gibt es keine Einschränkung, da sowohl die Fortbewegung von Regen, als auch das verschwinden gelernt werden soll. Jeder Zeitschritt wird maximal einmal in das Trainingsset aufgenommen, was einmal als Label verwendet wurde, wird keinesfalls in einem anderen Sample als als Eingabedatum verwendet. Beim Aufteilen des Sets zwischen Trainings und Validierungsset ist es wichtig, dass nicht zufällige Samples ausgewählt werden, damit garantiert ist, dass keine ähnliche Wetterlage bereits gesehen wurde.
Die so entstandenen Daten werden vor Eingabe in das Netz noch auf Werte zwischen 0 und 1 Normiert. Ab jetzt können die Daten für das Trainieren verwendet werden.



\subsection{Herausforderungen in diesem Kapitel}
Die größte Herausforderung in diesem Kapitel war zweifelsfrei die große Datenmenge, die wir verarbeitet haben. Die Rohdaten der 18 Jahre in 5-Minuten-Auflösung hätte unseren zugewiesenen Speicher gesprengt. Mit Btrfs, das die Dateien on-the-fly komprimiert und de-dupliziert, passten die Daten doch auf unsere Festplatte. Auch mussten wir eventuelle Ausreißer empirisch entfernen, weil das Berechnen des Histogramm über so viele Daten zu aufwendig wäre.

Darüber hinaus lagen die Archive des DWD im Tar-Format vor, das keine Checksumme anbietet und somit erst beim Entpacken bemerkt werden kann, dass beim Download einer Datei ein Fehler auftrat. Bei den großen Tar-Archiven des DWD ist leider mehrmals ein Download-Fehler aufgetreten, der sehr spät auffiel.


% Datenaufbereitung
\clearpage
\section{Daten-Aufbereitung}

\subsection{RADOLAN-Daten}

Die Daten des DWD werden über das Routineverfahren RADOLAN (Radar-Online-Aneichung) erfasst, dass durch eine Kombination von Niederschlagstationen und Wetterradaren hochauflösende Niederschlagsdaten produziert. 

Die Open-Source Bibliothek wradlib \footnote{\url{https://docs.wradlib.org/en/stable/index.html}} stellt für diese RADOLAN-Daten eine Stereographische Projektion zur Verfügung. Dies bedeutet das die Erdkugel zu einem Koordinatensystem aufgespannt wird, dessen Ursprung im Nordpol liegt. Siehe Abbildung  \ref{rz}.

\begin{figure}[H]
	\centering
	\includegraphics[width=0.8\textwidth]{pics/RZ_product.PNG}
	\caption{Ausschnitt Deutschlands im Koordinatensystem}
	\label{rz}
\end{figure}

Der resultierende Ausschnitt für Deutschland kann mit Hilfe von wradlib als 1100 * 900 Array ausgelesen werten, dessen Werte einer 1 Kilometer Grid-Box entsprechen.

\subsection{Location Konstanz}
Um für Konstanz und Umgebung sinnvolle Vorhersagen treffen zu können, muss der Standort von Konstanz auf den Daten markiert werden. Dazu müssen die Geographische Koordinaten für Konstanz in X- und Y-Koordinaten des Koordinatensystems umgewandelt werden. 

\begin{table}[H]
\begin{tabularx}{8cm}{|X|X|}
\hline
\multicolumn{2}{|l|}{ Geographische Koordinaten}   \\ \hline
         Latitude  & Longditude         \\ \hline
           47.66033 & 9.17582            \\ \hline
\end{tabularx} 	
\end{table}

\begin{table}[H]
\begin{tabularx}{8cm}{|X|X|}
\hline
\multicolumn{2}{|l|}{ XY - Koordinaten}   \\ \hline
        X-Koordinate & Y-Koordinate         \\ \hline
           -4602.6447 & -66.4622            \\ \hline
\end{tabularx} 	
\end{table}

Die umgewandelten XY-Koordinaten müssen nun innerhalb des Arrays gefunden werden. Dazu wurde über den Array iteriert und die Werte mit den errechneten Werten verglichen. Anschließend muss die Location auf der Karte markiert und dargestellt werden. Siehe Abbildung \ref{location}

\begin{figure}[H]
	\centering
	\includegraphics[width=0.8\textwidth]{pics/Location.png}
	\caption{Location von Konstanz im Ausschnitt Deutschland}
	\label{location}
\end{figure}

\subsection{Herausforderungen} 
Änderungen der Auflösung des Formats anpassen, richtige Datenquelle in wradlib finden. XY-Koordinaten innerhalb des Arrays finden und darstellen.

% Datenanalyse
\section{Datenanalyse}
Es wurden im Projekt unterschiedliche Daten auf ihre Quantität, Qualität und Verwendbarkeit untersucht. In diesem Schritt betrifft die Datenqualität erstmal die äußerlichen Merkmale, die tatsächliche Ähnlichkeit und der Vergleich von Regendaten folgt in einem späteren Kapitel. 
Für das Projekt werden zum einen die tatsächlichen Niederschlagsdaten in Konstanz zum späteren Vergleich benötigt, zum anderen Radardaten, beides möglichst in ein bis zehnminütlicher Auflösung und möglichst von aktuell bis einige Jahre in die Vergangenheit reichend. 
Wetterradardaten werden von öffentlichen und privaten Anbietern erhoben und sind in unterschiedlichen Qualitäten und unterschiedlicher Datenmenge verfügbar. Der E-Mail-Verkehr mit Meteogroup, einem privaten Wetterdienstleiter zeigte, dass die Daten der Wetterstationen Konstanz Südkurier und Konstanz Dettingen von uns nicht verwendet werden konnten. Die Daten von Windy sind zwar öffentlich zugänglich, jedoch nicht dowloadbar, welches eine Voraussetzung für verwendete Daten ist. Die Daten die Anbieters Kachelmann sind bis ins Jahr 2015 online verfügbar jedoch müssten hier die Nutzungsrechte eindeutig geklärt werden. 
Die umfangreichsten Radardaten werden von der Deutschen Wetterstation zur Verfügung gestellt, daher wurde sich im Team für die Verwendung dieser Daten entschieden. 

\subsection{Verifizierung der Daten mit statista und Wetterkontor}
In den Daten des DWD hat die Konstanzer Wetterstation die ID: 02712. Aus dem Zip-File mit der enthaltenen Textdatei können somit die nun aufgezeigten Daten für die Untersuchung entnommen werden. Zu den wichtigsten Informationen gehören hier das Qualitätsniveau der Daten (zwischen 1 und 16) als QN, das Messdatum im Format YYYYMMDDHHMM, die Niederschlagshöhe und den Niederschlagindikator. Die Aktuellen Daten wurden mit einem Python-Script eingelesen und ein paar kleine Statistiken zum Validieren der Daten vorgenommen. Zum Vergleich wurden hierfür die monatlichen Niederschlagsangaben von Statista.com und Wetterkontor.com genommen, allerdings bezogen die sich auf ganz Deutschland.
\begin{table}[ht]
\centering
\begin{tabular}{ll|l|ll}
\textbf{Monat} & \textbf{ref. $l/m^2$} & \textbf{Messung} & \textbf{Fehler (abs.)} & \textbf{Fehler (rel.)}\\\hline
Januar    & 40,1  & 37,39  & -2,71  & 6,76\%\\
Februar   & 34,5  & 37,02  & 2,52   & 7,30\%\\
März      & 40,2  & 40,3   & 0,1    & 0,25\%\\
April     & 132,4 & 132,25 & -0,15  & 0,11\%\\
Mai       & 56,4  & 52,58  & -3,82  & 6,77\%\\
Juni      & 95    & 69,55  & -25,45 & 26,79\%\\
Juli      & 168,9 & 171,38 & 2,48   & 1,47\%\\
August    & 155,7 & 148,42 & -7,28  & 4,68\%\\
September & 65    & 55,23  & -9,77  & 15,03\%\\
Oktober   & 36,3  & 36,93  & 0,63   & 1,74\%\\
November  & 76,8  & 78,49  & 3,69   & 4,93\%\\
Dezember  & 79,5  & 78,52  & -0,98  & 1,23\%\\
\end{tabular}
\caption{Auswertung des absoluten und relativen Fehlers zwischen den Regendaten verschiedener Quellen}
\label{tab:Auswertung}
\end{table}

Die Historischen Daten des DWD sind auch Monatlich als Textdatei gespeichert. Für den Vergleich wurde das Jahr 2018 betrachtet. In den historischen Daten fallen neue Spalten im Header auf, diese sind aber \gqq{ungenutzt} bzw. über alle (geprüft auf Jahr 2017) Einträge identisch. Diese sind: QN, RTH\_01 und RWH\_01. Die Messwerte wurden wie auch zuvor monatsweise zusammenaddiert das Ergebnis des Vergleichs ist in Tabelle~\ref{tab:Auswertung} zu sehen. 
Auffällig ist, dass einige Messwerte sehr gut übereinstimmen. So ist der Niederschlag in den Monaten März und April fast gleich in der referenzierten Messung und der Messung des DWDs. In anderen Monaten wie im Juni ist der relative Fehler mit 26.79\% sehr hoch. Das könnte an gemessenen Extremwerten in den Daten des DWDs liegen, die in der Größenordnung in den Messungen der anderen Anbieter nicht auftauchen.
Die hierzu verwendeten Skripte und Quellen sind auf GitHub wiederzufinden\footnote{\url{https://github.com/thgnaedi/DeepRain/tree/master/WetterStation_KN}}.  

\subsection{Korrelation der Radardaten mit den Regenmessungen}

In diesem Schritt sollte geprüft werden, wie hoch die gemessenen Radardaten mit den Niederschlagsdaten korrelieren. Im optimalen Fall sollten alleine auf Grundlage dieser Analyse Rückschlüsse auf die Lage von Konstanz auf dem Radarbildern durch eine besonders hohe Korrelation der Daten möglich sein. Als Monat für die Untersuchung wurde der Juni 2016 gewählt, da es sich um einen sehr Regenreichen Monat gehandelt hat.
Einen wichtigen Anteil an der Korrelationsanalyse hat die Datenaufbereitung in Anspruch genommen. Um einen Vergleich der Niederschlagsdaten mit den Radardaten zu ermöglichen ist es erforderlich, dass die Daten jeweils für dieselben Zeitabstände verfügbar sind. Da die Regenmessungen in minütiger Auflösung vorliegen, wurden sie mit Hilfe zweier Python-Skripte auf eine Datei mit 5-minütiger und eine mit stündlicher Auflösung angepasst. Nun wurde ein Notebook geschrieben, dass die Pixelwerte der Punkte um Konstanz also den Pixelwert 842.406 herum einliest und mit dem Zeitpunkt der Messung und dem tatsächlich gemessenen Niederschlag zu diesem Zeitpunkt in einer Numpy-Matrix speichert. Hier musste eine Einschränkung in Kauf genommen werden, da das einlesen jedes einzelnen Pixels bei 900 x 1100 Pixeln nicht möglich war und daher circa ein Umkreis von 20 Pixeln um Konstanz herum ausgewählt wurde. Bei der ersten Analyse wurde mit 156.377 ein falscher Pixel verwendet, was auf Unklarheiten bei der Achsenbestimmung zurückzuführen war. Die erstellte Numpy-Matrix wurde dann noch auf fehlende Werte untersucht, dieser Anteil hat sich als verschwindend gering herausgestellt, das heißt, dass die Daten bis auf ganz wenige Ausreißer vollständig sind. 
\begin{figure}[ht]
\centering
\includegraphics[width=\linewidth]{pics/plot_rainfall_day}
\caption{Regenfallmengen im Juni 2016 nach Tag}
\label{fig:Rainfall}
\end{figure}
Für die Analyse der Daten wurde dann ein R-Skript geschrieben. In diesem wurden die Datuminformationen extrahiert, sodass eine Einteilung in Jahr, Monat, Tag, Stunde und Minute der Datensätze möglich wurde. Dann wurden die die Regenmengen der gemessenen Niederschlagswerte je Tag dargestellt. Hier zeigte sich eine recht gleichmäßige Verteilung der Regenwerte mit einigen gut identifizierbaren Ausreißern z.B. am den 24. Juni mit über zehn Liter pro Quadratmeter, siehe \ref{fig:Rainfall}. Bei der Werteverteilung der Pixel aus den Radarbildern fällt auf, dass es sehr viele null-Werte und am zweitmeisten den Wert 80 gibt. Werte, die dazwischen existieren sind „10“, „20“, und „40“.  Bezüglich der Niederschlagsdaten könnte hier auch das eher niedrige Qualitätsniveau der Stufe 3 Einfluss gehabt haben.
Leider konnte zwischen den beiden Datentypen keine Korrelation festgestellt werden. Hierfür kommt eine Reihe von möglichen Gründen in Frage. Da die eigene Betrachtung der Daten, also der händische Abgleich von Zeiten mit Starken Regen in beiden Datengruppen nahelegt, dass die Daten tatsächlich nicht korrelieren kommt in Frage, dass die gemessenen Daten der Wetterstation Konstanz Fehlmessungen beinhalten. 

% Netzwerke
\section{Netzwerkarchitekturen}
%erster Kontakt MNIST wie viel Trefferrate mit einfacher architektur
Für den ersten Kontakt mit einem Neuronalen Netz befassten wir uns mit dem MNIST Datensatz. Hierbei geht es um eine Klassifizierung von handschriftlichen Zahlen in die zugehörigen zehn Klassen. Um Rechenzeit zu sparen, da uns anfangs noch keine GPU zur Verfügung stand, haben wir das Netz sehr klein gehalten. Das Netz für welches wir uns entschieden haben, besteht aus einem Convolutional Layer mit 32 3x3 Kerneln, diese werden im späteren Kapitel~\ref{kapitelCNN} genauer erläutert. Anschließend folgt ein 2x2 MaxPooling Layer, auch darauf gehen wir später noch ein. Um die Klassifikation vornehmen zu können, wird anschließen ein Flatten Layer eingebunden und ein Fully-Connected Layer mit 128 Output-Neuronen. Diese 128 Features werden dann über ein weiteres Fully-Connected Layer auf die 10 Ausgabeklassen verrechnet. Als Aktivierungsfunktion dient ein softmax, welcher dafür sorgt, dass die Summe aller Ausgaben eins ergibt. Somit können die Outputs als Wahrscheinlichkeiten gedeutet werden, die wahrscheinlichste Klasse wird dann als Vorhersage verwendet.
Das hier erwähnte Fully-Connected Layer ist die einfachste Art eines Layers, es besteht aus mehreren Eingabeneuronen und einigen Ausgabeneuronen. Jedes Ausgabeneuron ist hierbei mit jedem Eingabeneuron verbunden. Um den Wert eines Ausgabeneurons zu berechnen wird dann einfach eine gewichtete Summe aller Eingaben berechnet. Die hierbei verwendeten Gewichte werden über die Fehlerfunktion gelernt.

%Aufgabenstellung bzw. Lösung
Für die kurzzeitige Wettervorhersage entschieden wir uns für zwei grundlegende Aufgabenstellungen. Die erste Herangehensweise war das Vorhersagen weiterer Radarbilder in der Zukunft. Die Architektur muss also mehrere zusammengehörende Radarbilder als Eingabe verarbeiten und als Ausgabe wieder ein oder mehrere Zeitschritte liefern. Für diese Aufgabenstellung eignet sich sowohl ein klassisches CNN (Kaptel \ref{kapitelCNN}), als auch ein UNet(Kaptel~\ref{kapitelUNet}). Um uns zwischen diesen Architekturen zu entscheiden nahmen wir einen kurzen Test vor, in welchem beide Architekturen mittels MSE einen Zeitschritt (5 Minuten) vorhersagen sollten.
\begin{figure}[h]
	\centering
	\includegraphics[width=\linewidth]{pics/Syntetische_Daten_CNN_UNet.png}
	\caption[Lernkurven verschiedener Architekturen auf synthetischen Daten]{Gezeigt ist die Lernkurve der Validierungsdaten auf 100 Epochen mit unterschiedlichen Netzarchitekturen. Die verwendeten Trainingsdaten sind 1000 synthetische Bilder welche eine wandernde Regenfront simulieren sollen. Das CNN ist etwa gleich gut wie das einfache UNet. Eine tiefere UNet Architektur (grün) erreicht eine noch bessere Performance. Die Architekturen sind in nachfolgenden Kapiteln genauer erläutert. }
	\label{imgCNNUNet}
\end{figure}

Die Abbildung~\ref{imgCNNUNet} zeigt die Lernkurve der beiden Architekturen auf die identische Problemstellung. Das UNet lernt in diesem Beispiel deutlich besser, weshalb die Vorhersage von Radarbildern in Zukunft mit einem UNet behandelt wird. Beide UNet Architekturen haben zu Beginn eine deutlich schlechtere Vorhersage als das CNN, bereits nach 60 Epochen ist die einfache UNet Architektur gleich gut, während das grüne UNet die Performance sogar übertrifft und nach 100 Epochen die besten Ergebnisse liefert.
\newline
Die zweite Herangehensweise ist eine Klassifikation, hierbei geht es nicht darum, das exakte Radarbild vorherzusagen, sondern einzuordnen, ob es regnet oder nicht. Diese Aufgabe wurde als einfache Klassifikation für Konstanz, als auch als pixelweise Klassifikation, für alle eingehenden Pixel durchgeführt. Für die Aufgabe der Klassifikation jedes Pixels wurde wieder ein UNet verwendet. Bei der Aufgabe der einfachen Klassifikation für Konstanz kamen beide Architekturen zum Einsatz.

\subsection{CNN}
\label{kapitelCNN}
Ein Convolutional Neural Networks im klassischen Sinne ist ein Netzwerk mit mehreren Convolutional-Layer, häufig in Verbindung mit Pooling-Layer. Ein Convolutional-Layer besteht aus mehreren Filtern. Die Filter berechnen ein Output in Abhängigkeit mehrerer benachbarter 'Pixel'; die Größe der berücksichtigten Region hängt von der Filter- bzw. Kernelgröße ab.
Vorteile von Convolutional Layers sind zum einen, dass Nachbarschaften berücksichtigt werden, was gerade bei Bildern sehr sinnvoll ist, aber auch, dass der Speicherbedarf sehr gering ist, da nur eine kleine Anzahl an Gewichten für das komplette Bild verwendet werden müssen.
Ein Pooling-Layer reduziert die Feature-Größe indem jeweils nur das stärkste Signal einer Region weitergegeben werden.

%MSE:
Für die Vorhersage der Radardaten haben wir uns für ein sehr einfaches CNN (Convolutional Neural Network) entschieden.
Als Eingabe erwartet es mehrere Zeitschritte um darauf eine Vorhersage zu treffen. Der Input ist also dreidimensional wobei die dritte Dimension die Zeitschritte und die anderen Dimensionen die Bildauflösung beschreiben. Die Eingabe wird durch sechzehn Convolution Kernel der Größe 5x5 verrechnet. Anschließend folgen zweiunddreißig weitere 5x5 Kernel. Des weiteren ist ein optionales Dropout Layer eingebunden, welches nur zu Testzwecken verwendet wurde. Die Performance hatte sich dadurch aber nicht bemerkenswert verändert. Abschließend kommt ein Kernel, welcher die nun entstandenen Features zu einem Bild zusammenfasst. Hierzu wird ein 3x3 Kernel verwendet. Alle Layer sind mit einer ReLu als Aktivierungsfunktion ausgestattet.
Die Performance ist weniger gut, als ein UNet kann aber mit der sehr einfachen UNet Architektur mithalten. Zu sehen ist das in Abbildung \ref{imgCNNUNet} wo die blaue Kurve dem Lernverhalten der hier beschriebenen CNN Architektur entspricht.

%Klassifikation:
Für Klassifikation des Konstanz-Pixels entschlossen wir uns dazu, unser ursprüngliches CNN zum Lernen des MNIST Datensatz wiederzuverwenden. Dieses Netz besteht lediglich aus zweiunddreißig 3x3 Kerneln mit ReLu Aktivierungsfunktion. Anschließend folgt ein 2x2 MaxPooling und ein Flatten Layer. Die nun \gqq{flachen} Daten werden durch ein FullyConnected Layer auf 30 Neuronen reduziert. Darauf folgt ein Dropout Layer mit 20\% welches Overfitting verhindern soll. Abschließend folgt ein Layer, dass die Features auf drei Ausgabe-Neuronen verrechnet. Damit die Klassifizierung einfacher zu interpretieren ist, wird als Aktivierungsfunktion ein SoftMax verwendet.

\subsection{UNet}
\label{kapitelUNet}
%was ist ein Unet?
Ein Unet ist eine spezielle Form eines CNN. Es hat seinen Namen durch die 'U-Förmige' Architektur (siehe Abbildung~\ref{imgUNetA}).
\begin{figure}[h]
	\centering
	\includegraphics[width=\linewidth]{pics/UNet_Biomedical}
	\caption[UNet aus Paper von O. Ronneberger, P. Fischer und T. Brox]{Die hier abgebildete Architektur entstammt dem Paper zur biomedizinischen Bildsegmentierung \cite{DBLP:journals/corr/RonnebergerFB15}. Hier abgebildet ist eine typische UNet-Architektur, welche von den Eingabebildern (oben links) zu der Ausgabe (oben rechts) die Features stets verkleinert (rote Pfeile) und im späteren Verlauf auch wieder vergrößert (grüne Pfeile). Da bei dem wieder vergrößern der Daten nicht alle Informationen wiedergewonnen werden können, gibt es auch Querverbindungen (graue Pfeile) welche die zuvor errechneten Features an ein späteres Layer weitergibt.}
	\label{imgUNetA}
\end{figure}

Das Unet besteht aus mehreren Convolutional Layern, mit anschließendem Pooling. Hierdurch wird die Featuregröße immer weiter reduziert. Ab einem bestimmten Punkt wird die Featuregröße wieder aufgeblasen, um am Ende für den Output die gleiche Größe wie für den Input zu besitzen. Beim 'Aufblasen' kommt ein Upsampling Layer zum Einsatz. Da allerdings beim Vergrößern der Features nicht alle Informationen wiederhergestellt werden können, gibt es auch noch horizontale Verbindungen, durch welche zuvor erstellte Features später weiterverwendet werden können.
Diese Architektur ist sehr ähnlich zu einem Autoencoder und wurde erstmals für biomedizinische Zwecke verwendet. Da sich mit dieser Architektur mehrere Bilder einlesen und auch beliebig viele Bilder ausgeben lassen, versuchen wir damit die Wettervorhersage für mehrere Zeitschritte zu lösen.

%Einstiegsnetz
Um eine geeignete Architektur auszuwählen haben wir ein einfaches Testszenario aufgebaut, welches unterschiedliche Netzarchitekturen zu lösen hatten. Die hierfür verwendeten Daten waren aus jeweils 100x100x5 Pixeln großen synthetischen Daten aufgebaut.
%ToDo: Synthetische Daten erklären.
Das erste U-Net besteht aus lediglich einem Upsampling Layer. Die Architektur sieht wie folgt aus:
Anfangs verarbeiten zweiunddreißig 5x5 Convolutional Layer mit Padding die Eingabe. Diese wird über ein 3x3 MaxPooling reduziert. Die reduzierten Daten werden durch ein 3x3 Upsampling wieder in die ursprüngliche Größe zurück gewandelt und mit den Features vor dem MaxPooling verbunden. Die so entstehenden 100x100x64 Features werden anschließend über einen 1x1 Convolutional Layer zu einem Ausgabebild zusammengefasst.
Die zu diesem Netz gehörende Lernkurve ist in Abbildung~\ref{imgCNNUNet} zu sehen. Die Performance unterscheidet sich kaum zu der eines klassischen CNNs. Daher versuchen wir die Architektur tiefer zu gestalten.

%Unet64
Das finale UNet wie es in Abbildung \ref{imgCNNUNet} in grün zu sehen ist besteht aus mehreren Ebenden.
Zunächst wird die Eingabe durch ein Convolutional-Layer mit zehn 3x3 Kerneln und anschließender ReLU als Aktivierungsfunktion geleitet. Die so entstehenden Features werden als Conv01 bezeichnet und später wieder verwendet.
Die Features werden anschließend durch ein MaxPooling-Layer der Größe 2x2 verkleinert. Anschließend folgt ein weiteres Convolutional-Layer mit zwanzig 3x3 Kerneln. diese Features werden im weiteren als Conv02 bezeichnet.
Auch hierauf folgt wieder ein 2x2 MaxPooling-Layer mit anschließend einem weiteren Convolutional-Layer mit zwanzig Kerneln der Größe 3x3. Diese Features werden als Conv03 weiterverwendet.
Die tiefste Ebene in dieser Architektur besteht wieder aus einem Convolutional-Layer mit zwanzig Kerneln der Größe 3x3. Auch hier folgt ein 2x2 MaxPooling. Diese Features heißen Conv04.
Ab jetzt werden die Features wieder größer. Ein Upsampling-Layer der Größe 2x2 sorgt dafür, dass die Features mit den zuvor erstellten Conv04 Features verbunden werden können. Dies geschieht durch ein Concatenate-Layer, was die beiden Tensoren zu einem Größeren verbindet. Die so entstandenen Features werden wieder durch ein Upsampling-Layer der größe 2x2 vergrößert und mit den Conv03 Features verbunden. Auch diese Daten werden wieder über ein Upsampling-Layer der größe 2x2 verrechnet und anschließend mit den Conv02 Features verbunden. Abschließend werden die Daten durch ein weiteres 2x2 Upsampling-Layer verrechnet und mit den Conv01 Features verbunden. Abschließend kommt ein als Ausgabe ein Convolutional-Layer mit einem 1x1 Kernel der alle Features zu einem Graustufenbild zusammenfasst.
Alle Hier erwähnten Convolutional-Layer haben eine ReLU als Aktivierungsfunktion, wodurch negative Werte ausgeschlossen werden. Des weiteren besitzen diese Layer auch Padding um die Bildgröße beizubehalten.

Alle oben genannten Architekturen werden mit dem Adam-Optimizer trainiert. Je nach Aufgabenstellung dient als Fehlerfunktion der MSE, beim Vorhersagen der Radardaten, oder die Categorical Cross-Entropy, beim Lösen des Klassifikationsproblems.

\section{Regenradar Vorhersage}
Ziel dieser Aufgabe ist es, mit Hilfe eines UNets eine kurzzeit-Regenvorhersage zu erreichen. Die hier verwendeten Daten sind im Kapitel \ref{Samples} genauer erklärt. Ziel des Netzes ist es die Baseline zu schlagen. Als Baseline zählen wir die Vorhersage des zeitlich Letzten Bildes. Es wird bei der Baseline also davon ausgegangen, dass sich an dem Regenverhalten nichts ändert. Im ersten Unterkapitel wird das UNet verwendet um eine fünf-minuten Vorhersage zu machen. Es werden verschiedene Erweiterungen an dem bisherigen UNet angewandt um eine eventuelle performance Steigerung zu erreichen. Im zweiten Unterkapitel soll eine Vorhersage bis zu 35 Minuten in die ZUkunft möglich werden. Die Auswertung wird im jeweiligen Kapitel vorgenommen.

\subsection{Fünf Minuten vorhersage}
Um eine Radar-Vorhersage für fünf Minuten zu erreichen, werden Daten wie in Kapitel \ref{Samples} in Abbildung \ref{5D1L} gezeigt verwendet. Die Eingabedaten bestehen aus fünf aufeinander folgende Bilder, welche bis zu 20 Minuten in die Vergangenheit reichen. Das Radarbild im Label ist 5 Minuten in der Zukunft.

Die verwendete Architektur entspricht der des in Kapitel \ref{kapitelUNet} beschriebenen UNets. Die verwendete Fehlerfunktion ist der MSE. Das Training beschränkt sich auf 1900 Samples aus dem Jahr 2016, zum validieren wurden die letzten 50 Samples aus dem Dezember genommen. Die Validierungsdaten kommen somit in keiner Weise während des Trainings vor.
\begin{figure}[h]
	\centering
	\includegraphics[width=\linewidth]{pics/mse_learncurve.png}
	\caption[Lernkurve des UNets zur fünf Minuten Vorhersage]{Die dargestellte Lernkurve bezieht sich auf die fünf Minütige Radarvorhersage mittels MSE. Zum Trainieren wurden 1900 Samples aus dem Jahr 2016 verwendet. Abgebildet ist der Validation-Loss welcher auf weitere 50 Samples berechnet wurde. Die ersten Epochen ist eine sehr starke Verbesserung des Fehlers zu erkennen, auch nach 100 Epochen scheint die Fehlerkurve noch zu fallen.}
	\label{mseLC}
\end{figure}

Die erste Analyse der Lernkurve \ref{mseLC} zeigt, dass das Netz den Fehler (MSE) während der ersten 20 Epochen sehr stark verbessern kann. Auch bei der letzten Epoche scheint der Validation-Loss noch weiter zu fallen. Für dieses Einstiegsbeispiel soll das Ergebnis aber zunächst genügen. Es gilt noch zu klären, ob das Netz auch tatsächlich eine sinnvolle Vorhersage liefert und ob es in der Lage ist die Baseline zu schlagen. 

\begin{figure}[h]
	\centering
	\includegraphics[width=\linewidth]{pics/mse_vgl1.png}
	\caption[Radarbilder für fünf-minuten Vorhersage]{Hier abgebildet sind mehrere Radarbilder. Je heller ein Pixel dargestellt ist, desto stärker regnet es an dieser Stelle. Das hier als timestep 4 bezeichnete Bild ist das letzte Eingangs-Bild und spiegelt somit die Baseline wieder. Rechts daneben abgebildet ist das Label, welches fünf Minuten später repräsentiert. Ganz rechts ist die Vorhersage des Netzes zu sehen. Die eingezeichneten Linien dienen nur der Orientierung um ein Vergleich der Bilder sowie das schätzen der Bewegung der Regenfront zu ermöglichen. Es ist deutlich zu sehen, dass Die Vorhersage des Netzes stark verschwommen ist. Die Bewegung der Regenwolken wurde allerdings sehr gut vorhergesagt. }
	\label{mse_VGL1}
\end{figure}

Die Ausgabe sowie die Baseline sind in Abbildung \ref{mse_VGL1} für ein Validierungssample zu sehen. Das Netz hat die Bewegung der Regenfront gut vorhergesagt, allerdings ist die Regenintensität meist unterschätzt. Vor allem starken Niederschlag kann das Netz nicht vorhersagen. Das ist in der linken Abbildung an den weißen Punkten unten Links gut zu erkennen. Das Netz sagt hier eher einen dunkleren Grauton als das Label vorher. Um dennoch bewerten zu können, ob das Netz im Schnitt besser als die Baseline ist, wird für beide Vorhersagen der mittlere quadratische Fehler zum Label (im Bild 'timestep 5') berechnet. Das Resultat zeigt für die Baseline einen Fehler von 0.00165 und für das Netz etwa 0.0008. Die Baseline macht im Schnitt also einen doppelt so großen Fehler wie unser Netz. Diese Berechnung wurde für alle 50 Validierungssamples vorgenommen. Das Resultat ist in Abbildung \ref{vglBaseline1} zu sehen.

\begin{figure}[h]
	\centering
	\includegraphics[width=\linewidth]{pics/mse_baseline.png}
	\caption[Vergleich zwischen Baseline und Netzvorhersage.]{Auf der X-Achse sind die 50 Validierungssamples aufgelistet. Die Y-Achse entspricht dem MSE. Für jedes Sample wurde der MSE jeweils für die Vorhersage des Netzes (rot) und der Vorhersage der Baseline (grün) berechnet. Die Performance der Baseline konnte überboten werden.}
	\label{vglBaseline1}
\end{figure}

Einige Validierungsdaten konnten nahezu ohne Fehler vorhergesagt werden. Dies sind Bilder, bei welchen kein Regen mehr in dem Label zu sehen ist. Hat auch die Baseline keinen Fehler gemacht, war bereits im letzten Zeitschritt kein Regen mehr vorhanden. Alle anderen Samples beinhalten Regenwetter in den Labels. Hier ist die Baseline auch durchgehend schlechter. Dies kommt vor allem daher, dass sich Regenfronten sowohl verschieben, als auch die Regenmenge innerhalb fünf Minuten deutlich wechseln kann. selbst eine Verschiebung von lediglich einem Pixel resultiert in einem deutlichen Fehler.

Als nächsten Schritt wollen wir die Performance des Netzes noch weiter Steigern. Hierzu werden verschiedene Ansätze versucht. Die zugehörigen Lernkurven sind in Abbildung \ref{lc_unet_types} zu sehen.
Die bisher verwendete Architektur ist in Blau eingezeichnet. Der erste Versuch war es das Output-Layer statt mit einem 1x1 Kernel mit einem 3x3 Kernel auszustatten um regionale Einflüsse besser in das Ergebnis mit aufzunehmen. Das Ergebnis hiervon ist ein vor allem Anfangs deutlich schnelleres sinken des Fehlers. Nach 80 Epochen ist die Performance nur geringfügig besser. Ein weiterer Versuch ist das erweitern des Netzwerkes um eine neue Ebene. Diese Ebene besteht wieder aus 20 3x3 Kernel, das nun tiefere Netzwerk ist in Grün dargestellt und erreicht die selbe Performance wie das ursprüngliche Netz. eine deutliche Vergrößerung der neuen Schicht (von 20 auf 60 Kernel) kann die Performance minimal verbessern. Die zugehörige Lernkurve ist in Rot eingezeichnet. Ein anderer Ansatz ist das ersetzen der ReLU Aktivierungsfunktionen durch Sigmoid Aktivierungsfunktionen. Hierdurch erhöht sich der Rechenaufwand enorm, da die ReLU lediglich negative Werte abschneidet und eine Sigmoid-Funktion für jeden Wert einen neuen Funktionswert berechnen muss. Die Performance ist in Violett eingezeichnet und zeigt keine Verbesserung. Auch die Fehlerfunktion wird testweise ausgetauscht, statt dem MSE wird der MAE verwendet, welcher lediglich ohne Quadrate arbeitet. Aber auch hier (Braune Kurve) ist keine Verbesserung zu erkennen. Als letztes wird der Adam-Optimizer ausgetauscht. Die Alternative ist der Nestreov-Optimizer, welcher in der Lernkurve (Pink) deutliche Sprünge aufweist. Eine nennenswerte Änderung konnte allerdings mit keiner Optimierung erreicht werden. lediglich ein Unterschied in der Rechenzeit ist bemerkbar. Da die Netze zeitweise auch auf der CPU trainiert werden müssen, Da benötgte Hard- und Software nur teilweise während des Projekts zur Verfügung standen, wird weiterhin das Blaue bisher verwendete UNet verwendet.

\begin{figure}[h]
	\centering
	\includegraphics[width=\linewidth]{pics/vgl_lc_optim.png}
	\caption[Verschiedene UNet optimierungen im Vergleich.]{
		Um die Performance unserer UNet-Architektur weiter zu steigern wurden unterschiedliche Änderungen versucht. Die blaue Kurve zeigt das bisher verwendete Netz. Ein größerer Kernel am Output-Layer (gelb) verbessert die Performance ein wenig. Eine tiefere Netzwerkarchitektur (grün) steigert die Performance erst bei verwenden von sehr vielen Kernel (rot). Das Austauschen der Aktivierungsfunktionen gegen eine sigmoid-Funktion (violett) ändert außer an der Berechnungsgeschwindigkeit nichts. Das Austauschen der Fehlerfunktion gegen den MAE (braun) sowie das ändern des Optimierers zum Nesterov (pink) zeigen keine Verbesserung. Insgesamt werden nur deutliche Änderungen an der Rechenzeit festgestellt. Die Ergebnisse bleiben von der Performance vergleichbar.
		}
	\label{lc_unet_types}
\end{figure}

% -> Mehrere Zeitschritte vorhersagen
\subsection{Fünfunddreißig Minuten vorhersage}
Da die Baseline also mit dieser Architektur geschlagen werden kann, ist das nächste Ziel mehr als nur 5 Minuten vorherzusagen. Die hierfür verwendeten Daten bestehen aus 7 Zeitschritten im Label, also bis zu 35 Minuten in die Zukunft. Die Netzarchitektur muss hierfür lediglich am Output-Layer angepasst werden. Hierzu wird der 1x1 Kernel gegen sieben 1x1 Kernel ausgetauscht. Die Fehlerfunktion bleibt wie bisher der MSE, eine unterschiedliche Gewichtung der Zeitschritte wird nicht vorgenommen.
Für das Training wird nun auf alle zur Verfügung stehenden Daten zugegriffen. Dies entspricht 7500 Trainings- und 623 Validierungssamples.

\begin{figure}[h]
	\centering
	\includegraphics[width=\linewidth]{pics/lc_35minMSE.png}
	\caption[Lernkurve des UNet zur 35 Minuten Radar-Vorhersage.]{Die abgebildete Lernkurve zeigt, dass das Netz erst nach ca. 300 Epochen sich nicht weiter verbessert. Die rechte Abbildung ist ein Ausschnitt aus der linken, bei welcher das fallen der Lernkurve besser sichtbar ist.}
	\label{lc_35minMSE}
\end{figure}

Die zugehörige Lernkurve ist in Abbildung \ref{lc_35minMSE} zu sehen. Da die Performance auf das Validierungsset (blaue Kurve) nicht weiter verbessert wird, kann mit der Auswertung begonnen werden.
Zunächst wird das Netz wieder mit der Baseline verglichen. Hierfür wird jeder Zeitschritt aus dem Label mit dem letzten Zeitschritt der Eingabe-Daten verglichen. Da die Baseline davon ausgeht, dass das Wetter so bleibt wie es ist, ist zu erwarten, dass die Baseline mit zunehmender Zeit stets schlechter wird. Auch das Netz sollte mit zunehmender Zeit Probleme bekommen die Daten noch korrekt vorherzusagen. Eine direkter Vergleich pro Zeitschritt zwischen Baseline und Netzvorhersage wird in Abbildung \ref{35vglbasepred} dargestellt.
Die obere Zeile der Abbildung enthält die Verteilung der Vorhersage, während die untere Zeile die der Baseline enthält. Die Spalten geben die Zeitschritte an, zwei untereinander liegende Histogramme stellen die selbe Zeitliche Vorhersage dar. 
In den Histogrammen ist die Verteilung der Summe der quadratischen Fehler für alle Validierungsdaten dargestellt. Es wurde also pro Vorhersage eine Summe über alle quadratischen Abweichungen zwischen Vorhersage und Label, pro Zeitschritt, berechnet.
Schwarz eingezeichnet ist der Mittelwert, die beiden grauen Linien entsprechen dem 25\%-, bzw 75\%-Quantil. Es ist deutlich zu erkennen, dass der Mittelwert des Fehlers mit zunehmender Zeit größer wird, bzw. nach rechts wandert. Dies war so auch zu erwarten, da mit zunehmender Zeit stets neue Faktoren Einfluss auf das Wetterverhalten haben. Interessanter ist der Vergleich der Spalten. Der Mittelwert des Fehlers ist durchgehend besser bei der Vorhersage des Netzes als bei der Baseline. Mit zunehmender Zeit wird dieser Unterschied allerdings sehr gering.

\begin{figure}[h]
	\includegraphics[width=\linewidth]{pics/35min_vgl_baseline.png}
	\caption[Vergleich zwischen Baseline und UNet (35 Minuten Vorhersage)]{Ein Vergleich zwischen Baseline und Vorhersage. In den Spalten sind die Zeitlichen Schritte von 5 bis 35 Minuten Vorhersage. In der oberen Zeile ist die Vorhersage des Netzes und in der unteren die der Baseline abgebildet. Die Histogramme zeigen die Summe der Quadratischen Fehler pro Zeitschritt. Die schwarzen Linien zeigen den Mittelwert der Verteilung, während die grauen Linien das 25\%- sowie das 75\% Quantil darstellen. Mit zunehmender Zeit wandert die Verteilung weiter nach rechts, die Vorhersagen stimmen also weniger mit dem Label überein.}
	\label{35vglbasepred}
\end{figure}

Den Zahlen nach ist das Ziel die Baseline zu schlagen also geschafft. Um aber wirklich ein Fazit ziehen zu können werden die Daten und Vorhersagen in nachfolgender Abbildung \ref{vieleBilder1} und \ref{vieleBilder2} dargestellt.
Die hier verwendeten Daten und Label entstammen dem Validierungsset, welches Radardaten aus dem Jahr 2014 umfasst. sie wurden in keinem Trainingsschritt verwendet.
Die Abbildungen zeigen deutlich, dass das Netz die ersten fünf Minuten sehr genau vorhersagen kann. Das ausgegebene Bild hat unterschiedliche Niederschlagsmengen für jeden Pixel vorhergesagt. Mit zunehmender Zeit lässt diese Genauigkeit aber stark nach. So ist im letzten Ausgabebild sehr häufig für einen ganzen Bereich die selbe Niederschlagsmenge vorhergesagt. Es wirkt, als würde die Auflösung der Vorhersage mit zunehmender Zeit abnehmen. Durch die schwache Auflösung werden viele Pixel nicht ganz korrekt wiedergegeben, was den Fehler erhöht. Der so entstehende Fehler ist aber dennoch geringer, als der Fehler welchen die Baseline macht, indem sie exakte Niederschlagsmengen für jeden Pixel vorhersagt. Unser Netz schlägt die Baseline also nicht dadurch eine exaktere Vorhersage zu treffen, sondern durch eine verschwommene Regenfront, welche sich entsprechend der eingegangenen Daten weiter bewegt und dabei auch stets auseinander driftet.


\begin{figure}[ht]
	\centering
	\begin{subfigure}[]{\linewidth} 
		\centering
		\includegraphics[width=\linewidth]{pics/dt2.png}
		\caption[]{}\label{fa}
	\end{subfigure}
	
	\begin{subfigure}[]{\linewidth} 
		\centering
		\includegraphics[width=\linewidth]{pics/t2.png}
		\caption[]{}\label{fb}
	\end{subfigure}
	
	\caption[Validierungsdaten und Label sowie Vorhersage 1]{Die in (a) gezeigten Bilder entsprechen den eingehenden 25 Minuten an Radardaten. In (b) sind die 35 Minuten Label sowie darunter die 35 Minuten Vorhersagen abgebildet. Mit zunehmender Zeit wird die Vorhersage ungenauer. Während das erste Bild das Label noch gut wiedergeben kann, besteht das siebte nur noch aus sehr groben Strukturen.}
	\label{vieleBilder1}
\end{figure}

\begin{figure}[ht]
	\centering
	\begin{subfigure}[]{\linewidth} 
		\centering
		\includegraphics[width=\linewidth]{pics/dt5.png}
		\caption[]{}\label{ga}
	\end{subfigure}
	
	\begin{subfigure}[]{\linewidth} 
		\centering
		\includegraphics[width=\linewidth]{pics/t5.png}
		\caption[]{}\label{gb}
	\end{subfigure}
	
	\caption[Validierungsdaten und Label sowie Vorhersage 2]{Die in (a) gezeigten Bilder entsprechen den eingehenden 25 Minuten an Radardaten. In (b) sind die 35 Minuten Label sowie darunter die 35 Minuten Vorhersagen abgebildet. Im Gegensatz zu Abbildung \ref{vieleBilder1} kann hier anhand der oberen linken Ecke sehr gut eine Bewegung wahrgenommen werden. Obwohl die Regenfront erst in den letzten beiden Eingabebildern erscheint, kann sie in den sieben Ausgabebildern vorhergesagt werden. Zwar ist auch hier die Vorhersage sehr ungenau, aber die Bewegungsrichtung ist korrekt vorhergesagt.}
	\label{vieleBilder2}
\end{figure}


%wie gut zum nicht nass werden ?
% zum ende Kommen...
%ToDo: vgl mit Zeros?
%ToDo: Daten/Label erzeugung einfuegen -> \label{Samples}
%ToDo: Auch 5Daten1Label bild einbinden -> \label{5D1L}


\section{Klassifizierung}
Statt die genaue Regenmenge vorherzusagen, stellten wir drei Kategorien auf: kein Regen ($= 0mm$), wenig Regen ($\leq 8mm$) und viel Regen ($> 8mm$). Diese Kategorien haben wir als One-Hot-Vector kodiert. `[1, 0, 0]` entspricht hierbei kein Regen, sodass man aus der ersten Dimension der Vorhersage einfach ein Vorschaubild generieren kann aus dem man gleich feststellen kann, ob es am jeweiligen Pixel regnet oder nicht.

Für das Training mit Kategorien kann man nicht mehr den MSE verwenden, hier würde selbst nach 80 Epochen nur "kein Regen" vorhergesagt. Stattdessen wurde als Loss-Funktion die \enquote{Categorical Crossentropy} von Keras verwendet; die binäre Crossentropy können wir nicht verwenden, weil wir mehr als zwei Kategorien verwenden. Die \enquote{Categorical Crossentropy} funktioniert relativ gut, aber es wird ein Blob vorhersagt, der etwas über den Bereich ragt, in dem es eigentlich regnet.

Danach wurde noch die Aktivierungsfunktion für den Output-Layer Sigmoid durch Softmax ersetzt. Dadurch erscheint das Vorschaubild etwas verwaschener, aber der Blob um das Regengebiet wird kleiner und die Differenz zum Referenzbild wird kleiner.

Wenn man die Aktivierungsfunktion der Hidden-Layer (von ReLu) zu Tanh verändert, verbessert sich auch die Kategorisierung: der Blob nähert sich weiter dem Regengebiet aus dem zu vorhersagendem Bild an, ist aber immer noch merkbar größer und franst an den kanten aus.

Als nächstes wird die Metrik "categorical\_accuracy" verwendet, um die Vorhersage zu überwachen. Dadurch kann der Fortschritt beim Trainieren besser überwacht werden.


\subsection{Training}
Für das Training wurden alle Daten der 18 Jahre verwendet, partitioniert in Trainings- und Evaluationsdaten. Als Lossfunktion wurde der MSE verwendet, der den Unterschied über alle Pixel kumuliert, weswegen so hohe Werte in den Abbildungen vorkommen (1100x900 Pixel). Da das Training auf der GPU weniger al 10~Sekunden pro Epoche dauert, wurden gleich 3072 Epochen trainiert.

In Abbildung~\ref{fig:lernkurven} sind die Lernkurven des Trainings nebeneinander dargestellt, links steht das Training des Netzes mit Softmax als Aktivierungsfunktion des Hidden Layer und rechts ist das selbe Netz mit TanH als Aktivierungsfunktion. Man sieht jedoch, dass es ab etwa 1500~Epochen (x-Achse) keine Verbesserungen mehr gibt.

\begin{figure}[ht]
\centering
\begin{subfigure}{0.5\textwidth}
\centering
\includegraphics[width=\linewidth]{pics/lernkurve_activationHidden-softmax_activationOutput-softmax}
\caption{Lernkurve (Hidden layer: Softmax)}
\label{fig:lernkurveSoftmax}
\end{subfigure}%
\begin{subfigure}{0.5\textwidth}
\centering
\includegraphics[width=\linewidth]{pics/lernkurve_activationHidden-tanh_activationOutput-softmax}
\caption{Lernkurve (Hidden layer: Tanh)}
\label{fig:lernkurveTanh}
\end{subfigure}%
\caption{Lernkurven von den Aktivierungsfunktionen der Hidden-Layer}
\label{fig:lernkurven}
\end{figure}



\subsection{Auswertung}
Zu den Netzen mit den verschiedenen Aktivierungsfunktionen wurden jeweils eine Confustion-Matrix stellt (siehe Tabellen~\ref{tab:confusionSoftmax} und~\ref{tab:confusionTanh}). An beiden Matrizen kann man sehen, dass beide Aktivierungsfunktionen für einzelne Pixel statistisch ähnliche Ergebnisse liefern.

Bei beiden Matrizen ist gleich, dass, falls kein Regen vorhergesagt wird, in 96,1\% respektive 96,3\% auch tatsächlich kein Regen eintrifft. Wird wenig Regen vorhergesagt, ist die Unsicherheit recht groß: zu rund 20\% regnet es gar nicht oder zu etwa 16\% regnet es stark. Zu einem Fünftel kann es also sein, dass hier die Vorhersage \enquote{Regen} nicht eintrifft. Wird starker Regen vorausgesagt, trifft zu gut 68\% auch starker Regen ein, zu fast 26\% wenig Regen oder gut 5\% kein Regen. Hier trifft die Vorhersage \enquote{Regen} also zu 95\% ein.

\begin{table}[ht]
\centering
\begin{tabular}{lr|rrr}
    &                      & \multicolumn{3}{c}{Vorhersage}\\
    &                      & \textbf{Kein Regen} & \textbf{Wenig Regen} & \textbf{Viel Regen}\\\hline
\multirow{3}{*}{\rotatebox{90}{Daten}}
    & \textbf{Kein Regen}  & 2227245 (96,1\%)    & 33383 (19,9\%)       & 3762 (05,7\%)\\
    & \textbf{Wenig Regen} & 81116 (03,5\%)      & 106434 (63,4\%)      & 17077 (25,9\%)\\
    & \textbf{Viel Regen}  & 9695 (00,4\%)       & 27930 (16,7\%)       & 45166 (68,4\%)\\
\end{tabular}
\caption{Confustion-Matrix (Aktivierungsfunktion Hidden Layer: Softmax)}
\label{tab:confusionSoftmax}
\end{table}

\begin{table}[ht]
\centering
\begin{tabular}{lr|rrr}
    &                      & \multicolumn{3}{c}{Vorhersage}\\
    &                      & \textbf{Kein Regen} & \textbf{Wenig Regen} & \textbf{Viel Regen}\\\hline
\multirow{3}{*}{\rotatebox{90}{Daten}}
    & \textbf{Kein Regen}  & 2225229 (96,3\%)    & 35527 (20,4\%)       & 3634 (05,4\%)\\
    & \textbf{Wenig Regen} & 76988 (03,3\%)      & 110399 (63,5\%)      & 17240 (25,8\%)\\
    & \textbf{Viel Regen}  & 8849 (00,4\%)       & 27969 (16,1\%)       & 45973 (68,8\%)\\
\end{tabular}
\caption{Confustion-Matrix (Aktivierungsfunktion Hidden Layer: Tanh)}
\label{tab:confusionTanh}
\end{table}

Bei der Vorhersage von zusammenhängenden Niederschlagsmengen auf einer Karte gibt es Unterschiede zwischen den Aktivierungsfunktionen der Hidden Layer, während es auf die Wahrscheinlichkeit der einzelnen Pixel kaum unterschiede gab.
Das UNet mit TanH als Aktivierungsfunktion der Hidden Layer erzeugt größere (und leicht rundere) Blobs (Abbildung~\ref{fig:hiddenActivationTanh}). Bei Softmax werden schräge Kanten besser vorhergesagt (Abbildung~\ref{fig:hiddenActivationSoftmax}).
\begin{figure}[ht]
\centering
\begin{subfigure}{0.5\textwidth}
\centering
\includegraphics[width=\linewidth]{pics/categorical_crossentropy_hidden-softmax_output-softmax_above_3072}
\caption{Hidden layer activation: Softmax}
\label{fig:hiddenActivationSoftmax}
\end{subfigure}%
\begin{subfigure}{0.5\textwidth}
\centering
\includegraphics[width=\linewidth]{pics/categorical_crossentropy_hidden-tanh_output-softmax_above_3072}
\caption{Hidden layer activation: Tanh}
\label{fig:hiddenActivationTanh}
\end{subfigure}%
\caption{Vergleich von Aktivierungsfunktionen der Hidden-Layer}
\label{fig:activatinHidden}
\end{figure}


\subsubsection{Zwei Kategorien}
Daraufhin versuchten wir die Vorhersage mit nur zwei Kategorien (Regen / Kein Regen) durchzuführen um festzustellen, ob die Vorhersage ob es regnet oder nicht besser funktioniert als die mit keinem, wenig oder starkem Regen. Dazu erstellen noch einmal die Confusion Matrix auf (siehe Tabelle~\ref{tab:confusionTwoCategoriesTresholdZero}). Bei einem Treshold von 0 erhält man fast gleiche Wahrscheinlichkeiten wie bei 3 Kategorien für Regen und Kein Regen: Falls kein Regen vorhergesagt wurde, regnet es zu 96\% nicht; und falls Regen vorhergesagt wurde regnet es zu gut 80\%.

\begin{table}[ht]
\centering
\begin{tabular}{lr|rr}
    &                      & \multicolumn{2}{c}{Vorhersage}\\
    &                      & \textbf{Kein Regen} & \textbf{Regen}\\\hline
\multirow{3}{*}{\rotatebox{90}{Daten}}
    & \textbf{Kein Regen}  & 2218754 (96,3\%)    & 45636 (18,4\%)\\
    & \textbf{Regen}       & 85427 (03,7\%)      & 201991 (81,6\%)\\
\end{tabular}
\caption{Confustion-Matrix (Zwei Kategorien, Treshold: 0)}
\label{tab:confusionTwoCategoriesTresholdZero}
\end{table}

\begin{table}[ht]
\centering
\begin{tabular}{lr|rr}
    &                      & \multicolumn{2}{c}{Vorhersage}\\
    &                      & \textbf{Kein Regen} & \textbf{Regen}\\\hline
\multirow{3}{*}{\rotatebox{90}{Daten}}
    & \textbf{Kein Regen}  & 2272947 (96,7\%)    & 39276 (19,6\%)\\
    & \textbf{Regen}       & 78475 (03,3\%)      & 161110 (80,4\%)\\
\end{tabular}
\caption{Confustion-Matrix (Zwei Kategorien, Treshold: 2)}
\label{tab:confusionTwoCategoriesTresholdTwo}
\end{table}



\subsection{Herausforderungen in diesem Kapitel}
\begin{itemize}
\item Richtige Kategorien finden
\item Training mit richtiger Aktivierungsfunktion / Optimizer
\end{itemize}


% Ziel und Test: MSE

% Fazit / Ausblick

\newpage
\listoffigures
\listoftables

\bibliography{paper/bib/report.bib}
\bibliographystyle{alpha}

\end{document}
