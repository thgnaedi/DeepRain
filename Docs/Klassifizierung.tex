\section{Klassifizierung}
Statt die genaue Regenmenge vorherzusagen, stellten wir drei Kategorien auf: kein Regen (= 0mm), wenig Regen (<= 8mm) und viel Regen (> 8mm). Diese Kategorien haben wir als One-Hot-Vector kodiert. `[1, 0, 0]` entspricht hierbei kein Regen, sodass man aus der ersten Dimension der Vorhersage einfach ein Vorschaubild generieren kann aus dem man gleich feststellen kann, ob es am jeweiligen Pixel regnet oder nicht.

Für das Training mit Kategorien kann man nicht mehr den MSE verwenden, hier würde selbst nach 80 Epochen nur "kein Regen" vorhergesagt. Stattdessen wurde als Loss-Funktion die \enquote{Categorical Crossentropy} von Keras verwendet; die binäre Crossentropy können wir nicht verwenden, weil wir mehr als zwei Kategorien verwenden. Die \enquote{Categorical Crossentropy} funktioniert relativ gut, aber es wird ein Blob vorhersagt, der etwas über den Bereich ragt, in dem es eigentlich regnet.

Danach wurde noch die Aktivierungsfunktion für den Output-Layer Sigmoid durch Softmax ersetzt. Dadurch erscheint das Vorschaubild etwas verwaschener, aber der Blob um das Regengebiet wird kleiner und die Differenz zum Referenzbild wird kleiner.

Wenn man die Aktivierungsfunktion der Hidden-Layer (von ReLu) zu Tanh verändert, verbessert sich auch die Kategorisierung: der Blob nähert sich weiter dem Regengebiet aus derm zu vorhersagendem Bild an, ist aber immer noch merkbar größer und franst an den kanten aus.

Als nächstes wird die Metrik "categorical\_accuracy" verwendet, um die Vorhersage zu überwachen. Dadurch kann der Fortschritt beim Trainieren besser überwacht werden.

\begin{table}[ht]
\begin{tabular}{ll|rrr}
                                     &                      & \multicolumn{3}{c}{Vorhersage}\\
                                     &                      & \textbf{Kein Regen}    & \textbf{Wenig Regen}    & \textbf{Viel Regen}\\\hline
\multirow{3}{*}{\rotatebox{90}{Echt}}& \textbf{Kein Regen}  & 2225229                & 35527                   & 3634\\
                                     & \textbf{Wenig Regen} & 76988                  & 110399                  & 17240\\
                                     & \textbf{Viel Regen}  & 8849                   & 27969                   & 45973\\
\end{tabular}
\caption{Confustion-Matrix (Aktivierungsfunktion Hidden Layer: Tanh)}
\label{tab:confusionTanh}
\end{table}

\begin{figure}[ht]
\centering
\begin{subfigure}{0.5\textwidth}
\centering
\includegraphics[width=\linewidth]{pics/categorical_crossentropy_hidden-softmax_output-softmax_above_3072}
\caption{Hidden layer activation: Softmax}
\label{fig:hiddenActivationSoftmax}
\end{subfigure}%
\begin{subfigure}{0.5\textwidth}
\centering
\includegraphics[width=\linewidth]{pics/categorical_crossentropy_hidden-tanh_output-softmax_above_3072}
\caption{Hidden layer activation: Tanh}
\label{fig:hiddenActivationTanh}
\end{subfigure}%
\caption{Vergleich von Aktivierungsfunktionen der Hidden-Layer}
\label{fig:activatinHidden}
\end{figure}

\subsection{Herausforderungen in diesem Kapitel}
\begin{itemize}
\item Richtige Kategorien finden
\item Training mit richtiger Aktivierungsfunktion / Optimizer
\end{itemize}
